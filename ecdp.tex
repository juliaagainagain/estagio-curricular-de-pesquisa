% ----------------------------------------------------------
% Modelo ABNT completo — Trabalho de Pesquisa
% ----------------------------------------------------------
\documentclass[
12pt,                
oneside,             
a4paper,             
chapter=TITLE,       
english,             
brazil               
]{abntex2}

% ----------------------------------------------------------
% PACOTES BÁSICOS
% ----------------------------------------------------------
\usepackage[utf8]{inputenc}      
\usepackage[T1]{fontenc}         
\usepackage{lmodern}             
\usepackage{indentfirst}         
\usepackage{graphicx}            
\usepackage{microtype}           
\usepackage{booktabs}            
\usepackage{array}
\usepackage{setspace}
\usepackage{hyperref}
\usepackage{float}
\usepackage{lipsum}
\usepackage{pgfplots}
\pgfplotsset{compat=1.18}

% ----------------------------------------------------------
% CONFIGURAÇÕES DE DOCUMENTO
% ----------------------------------------------------------
\setlength{\parindent}{1.25cm}
\setlength{\parskip}{0.2cm}
\OnehalfSpacing

\titulo{Impactos econômicos do turismo literário de “Drácula” no vilarejo de Bran, na Romênia}
\autor{Júlia Michetti Costa Santos\\ Fernanda França Silva}
\local{Belo Horizonte}
\data{2025}
\tipotrabalho{Trabalho de Pesquisa — Estágio Curricular}
\preambulo{%
Trabalho apresentado à disciplina \textit{Estágio Curricular de Pesquisa} do curso de graduação em Turismo da Universidade Federal de Minas Gerais.}

% ----------------------------------------------------------
% REMOVE ORIENTADOR E INSTITUIÇÃO DA FOLHA DE ROSTO
% ----------------------------------------------------------
\makeatletter
\renewcommand{\imprimirorientador}{}
\renewcommand{\imprimirinstituicao}{}
\makeatother

% ----------------------------------------------------------
% INÍCIO DO DOCUMENTO
% ----------------------------------------------------------
\begin{document}
\selectlanguage{brazil}
\frenchspacing

% CAPA
\imprimircapa

% FOLHA DE ROSTO
\imprimirfolhaderosto*

% ----------------------------------------------------------
% ELEMENTOS PRÉ-TEXTUAIS
% ----------------------------------------------------------

\begin{resumo}
O presente trabalho busca compreender os impactos econômicos do turismo literário relacionado à obra “Drácula”, de Bram Stoker, no vilarejo de Bran, na Romênia. O estudo pretende examinar como o mito do vampiro influencia o fluxo turístico, a geração de renda e o desenvolvimento local, a partir de uma abordagem interdisciplinar entre turismo, literatura e economia. 

\vspace{\onelineskip}
\noindent
\textbf{Palavras-chave}: turismo literário; economia do turismo; Drácula; Romênia; Bran.
\end{resumo}



% ----------------------------------------------------------
% Sumario
% ----------------------------------------------------------
\tableofcontents % SUMÁRIO automático

% ----------------------------------------------------------
% ELEMENTOS TEXTUAIS
% ----------------------------------------------------------
\textual

\newpage
\chapter{Introdução}
De acordo com Butler (2012), o turismo literário se caracteriza pela motivação de visitar lugares que mantêm alguma relação com a literatura, sejam espaços que inspiraram narrativas, residências de autores ou locais associados a personagens e enredos ficcionais. Nessa perspectiva, a literatura ultrapassa seu papel artístico e passa a influenciar a materialidade dos lugares, transformando o imaginário em destino turístico. 

O Castelo de Bran, na Romênia, representa um exemplo emblemático desse fenômeno: Na história, o personagem Jonathan Harker, um jovem advogado inglês, chega à Transilvânia para tratar de negócios, mas logo se vê preso em um castelo envolto em mistérios. Atordoado pelo ambiente, Harker torna-se refém de sua própria trajetória, registrando em cartas e documentos a experiência que atravessa o limiar entre o real e o imaginário. De forma análoga, os visitantes contemporâneos percorrem o castelo de Bran, movidos pela curiosidade e pela fantasia criada por Bram Stoker, buscando reviver, a aventura do personagem. Como observa Stijn Reijnders em Stalking the Count: Dracula, Fandom e Tourism (2011), esses turistas estabelecem uma conexão emocional e imaginativa com os locais associados à obra, transformando a visita em uma experiência estética e narrativa. Inserido no conceito de dark tourism, esse fenômeno revela como o fascínio pelo sombrio, pelo misterioso e pelo imaginário literário permite que a ficção de Stoker se transforme em vivência, criando um diálogo entre história, literatura e experiência sensorial.


Como observa Frost (2016, p. 45), “o vampiro criado por Bram Stoker e o mito de Drácula influenciaram diretamente a maneira como o governo romeno passou a trabalhar o turismo, principalmente para demandas ocidentais”. Nessa mesma linha, pesquisas em literatura e turismo, como as de Baleiro e Quinteiro (2018), evidenciam que a relação entre texto literário e território ultrapassa o campo simbólico e configura-se como uma área de estudo interdisciplinar, na qual se entrelaçam cultura, memória e economia.

A importância de mensurar esses efeitos na economia é destacada por Dwyer, Forsyth e Spurr (2004, p. 308), ao afirmarem que “a análise dos impactos econômicos do turismo oferece a evidência mais tangível e convincente da contribuição do setor para o desenvolvimento regional”. 

Essa perspectiva reforça a necessidade de observar o fenômeno do turismo literário não apenas em sua dimensão simbólica, mas também como motor de crescimento. No caso do vilarejo de Bran, o mito de Drácula consolidou o castelo como principal atrativo da região e como um dos destinos mais visitados da Romênia, influenciando diretamente o desenvolvimento econômico local, e que de acordo com Fletcher et al. (2018) ressalta que a mensuração dos impactos financeiros do turismo é um dos instrumentos mais concretos para compreender a evolução de um destino, pois reflete diretamente sua capacidade de gerar receita, empregos e investimentos locais fator que ainda é pouco estudado embora seja o meio mais tangível de medir a evolução de turismo de um local.

Esse estudo se propoe a compreender a forma que o turismo literário associado à obra Drácula impacta a economia do vilarejo de Bran, na Romênia, é fundamental para analisar as transformações econômicas decorrentes da atividade turística na região.












% ----------------------------------------------------------
% Referencial teorico
% ----------------------------------------------------------
\chapter{Referencial Teorico}





%turismo literario
\section{Entre páginas: O despertar turistico de Drácula}

%o vilarejo de bran
\section{Bran: O coração sombrio da Romênia}





%Impactos econômicos e indicadores
\section{Economia:}

Para compreender a economia do turismo literário, é necessário remontar às origens do pensamento econômico moderno. Adam Smith (1776) concebeu a economia como o estudo da produção, distribuição e consumo da riqueza das nações, identificando o trabalho humano como a força vital dessa riqueza. Sua visão inaugural permanece relevante, servindo como uma base conceitual para a análise contemporânea:

\begin{quote}
“O trabalho anual de cada nação é o fundo que originalmente a supre com todas as necessidades e conveniências da vida que anualmente consome, e que consistem sempre, ou no produto imediato desse trabalho, ou no que é comprado com esse produto de outras nações.” (SMITH, 1776, p. 10)
\end{quote}

Dentro desse prisma, o turismo literário pode ser compreendido como um esforço coletivo, no qual guias, escritores, livreiros, organizadores de festivais e artesãos criam experiências que, como castelos simbólicos, preservam a memória de nomes e narrativas. A experiência do visitante corresponde ao consumo dessa riqueza. No caso da cidade de Bran, o turismo literário associado à obra \textit{Drácula} impulsionou o desenvolvimento de indicadores econômicos capazes de traduzir experiências subjetivas em dados quantificáveis.

\section{Dados como a sombra sob a luz da economia.}

A caracterização de uma atividade econômica como pertencente ao setor turístico ocorre quando sua produção principal é composta por bens e serviços cuja demanda é fortemente influenciada pelos visitantes (IBGE, 2006). Diferentemente de outros setores econômicos, cuja mensuração é direta, o turismo é definido a partir da demanda, ou seja, do consumo realizado por não-residentes. É nesse contexto que surge o conceito de \textit{Atividades Características do Turismo} (ACTs), uma ferramenta para isolar e quantificar o impacto econômico do turismo. Embora muitas ACTs, como serviços de alimentação e transporte, também sejam consumidas por residentes, sua inclusão se justifica pela influência significativa da demanda turística, permitindo análises mais precisas sobre a contribuição do setor para o Produto Interno Bruto (PIB) (Ribeiro et al., \cite{ribeiro2021}).

\begin{table}[htbp]
\centering
\caption{Atividades Características do Turismo (IBGE)}
\label{tab:atividades_turismo}
\begin{tabular}{l}
\hline
\textbf{Atividade Característica do Turismo} \\ \hline
Hotéis, pousadas e similares \\
Restaurantes, bares e similares \\
Transporte rodoviário de passageiros \\
Transporte marítimo de passageiros \\
Transporte aéreo de passageiros \\
Serviços auxiliares ao transporte de passageiros \\
Agências de viagens, operadoras e similares \\
Aluguel de veículos e equipamentos de transporte de passageiros \\
Serviços esportivos e de lazer \\ \hline
\end{tabular}
\begin{flushleft}
\footnotesize{Fonte: Instituto Brasileiro de Geografia e Estatística (IBGE), \textit{Economia do turismo: análise das atividades características do turismo – 2003}, p. 14. Adaptado.}
\end{flushleft}
\end{table}


Conceituanddo a tabela o gasto com hospedagem constitui um dos principais componentes do consumo turístico, refletindo a necessidade básica de pernoite do visitante. Entre as atividades características do turismo (ACTs), os serviços de alojamento e alimentação representam pilares centrais da experiência turística e da economia do setor (IBGE). 

A relação dessas atividades com o turismo se evidencia na sazonalidade e na localização geográfica: estabelecimentos situados em áreas de grande fluxo turístico dependem intrinsecamente dessa demanda, e sua expansão ou retração é um indicador direto da saúde do setor.

 
 No presente contexto, o vinho Château Bran Cabernet Sauvignon exemplifica como produtos regionais podem ampliar a experiência do visitante: mais do que um consumo local, o vinho funciona como uma extensão da visita ao patrimônio. Produzido na região vinícola de Dealu Mare e comercializado na loja oficial do Castelo de Bran, ele permite que os turistas vivenciem a cultura local, ao mesmo tempo em que fortalece a economia regional e promove o turismo temático.
 
\begin{figure}[H]
    \centering
    \caption{\centering Vinho Château Bran Cabernet Sauvignon}
    \includegraphics[width=0.4\linewidth]{vinho.png}
    
    \label{fig:vinho_chateau_bran}
    \vspace{2mm}
    {\centering \footnotesize Fonte: CASTELUL BRAN, 2025. Disponível em: \url{https://brancastleshop.com/en/product/vin-cabernet-sauvignon-chateau-bran/}. Acesso em: 28 out. 2025.\par}
\end{figure}


Os serviços de transporte e a intermediação de serviços turísticos constituem a infraestrutura essencial para o acesso e a organização da viagem. O transporte conecta o turista ao destino, sendo um pré-requisito para a atividade turística, enquanto as agências de viagens agregam e facilitam serviços, funcionando como mediadoras da experiência. A saúde financeira dessas ACTs de infraestrutura e intermediação indica a conectividade e a capacidade de distribuição do produto turístico em um país ou região.

Complementando a experiência do visitante, os serviços desportivos e de lazer (incluindo atividades culturais, parques temáticos, entre outros) são frequentemente os motivadores finais da viagem. A relevância dessas atividades está em sua capacidade de transformar a visita em uma experiência memorável, gerando receita e emprego em setores além do básico (alojamento e alimentação), demonstrando a ampla comunhão do turismo na economia.

Como a organização desses dados e de suma importacia a OMT padroniza o dados sobre as ACTS sendo elas trabalhadas amplamente pelo mundo como forma de mensurar o impacto na região, como uma forma de esclarecer com os dados os impactos economicos.


VOU DEFINIR MACRO E MICRO ECONOMIA 














% ----------------------------------------------------------
% metodologia
% ----------------------------------------------------------

\chapter{Metodologia}

O presente estudo busca compreender de que forma o turismo literário associado à obra \textit{Drácula}, de Bram Stoker, impacta a economia do vilarejo de Bran, na Romênia. A partir dessa problemática, estabelecem-se as seguintes questões centrais: (1) quais são os principais impactos econômicos diretos e indiretos do turismo literário de \textit{Drácula} em Bran; (2) como o mito literário influencia o fluxo turístico e a geração de renda local; (3) de que maneira os setores característicos do turismo  (como hospedagem, alimentação, transporte e lazer)  são afetados; e (4) como o caso de Bran se compara a outros destinos literários internacionais.  

A pesquisa adota uma \textbf{abordagem mista}, combinando métodos \textbf{qualitativos e quantitativos}, para integrar dimensões simbólicas e econômicas do turismo literário. De natureza exploratória e descritiva, esta investigação estrutura-se como um \textbf{estudo de caso}. A base teórico-metodológica segue o que é proposto por Lima Junior, Oliveira, Santos e Schnekenberg (2021), ao apresentarem a \textbf{Análise Documental como percurso metodológico na pesquisa qualitativa}, enfatizando a importância da interpretação de documentos diversos para compreender fenômenos sociais e culturais.  

.  

\section{Pesquisa bibliográfica}

A revisão bibliográfica constitui a base teórica da pesquisa e será conduzida por meio da análise crítica de literatura científica, abrangendo temas que possibilitem compreender o turismo literário sob perspectivas cultural, social e econômica. Foram utilizadas as bases \textit{Google Scholar}, \textit{CAPES}, \textit{SciELO} e a plataforma \textit{Sucupira} para a verificação de qualidade das fontes.  

Os principais eixos teóricos que fundamentam esta pesquisa abrangem quatro dimensões centrais. A primeira refere-se ao turismo literário e aos seus fundamentos culturais, compreendidos como expressão da relação entre narrativa, identidade e território. A segunda aborda os impactos econômicos do turismo, com ênfase nas metodologias de mensuração que permitem avaliar a contribuição dessa atividade para o desenvolvimento local. A terceira dimensão trata da Análise Documental como percurso metodológico, sustentada pelos pressupostos de Sá-Silva, Almeida e Guindani (2009), Cellard (2008) e Lüdke e André (1986), que destacam a relevância desse método para a investigação de fenômenos sociais e culturais a partir de registros e fontes documentais. Por fim, a quarta dimensão considera a pesquisa qualitativa e a triangulação metodológica, conforme as orientações de Minayo (2009) e de Tuzzo e Braga (2016), que defendem a integração entre diferentes técnicas de análise como meio de ampliar a validade e a profundidade interpretativa dos resultados. Esses referenciais sustentam a compreensão do turismo literário como prática social e cultural e orientam a análise dos documentos oficiais e dados secundários utilizados no estudo.  

\section{Análise documental}

A segunda etapa da metodologia consiste na \textbf{Análise Documental}, fundamentada nos conceitos de Cellard (2008) e Sá-Silva et al. (2009), que definem esse método como o exame sistemático de documentos de diferentes naturezas, a fim de extrair informações significativas relacionadas ao objeto de estudo. Conforme Lima Junior et al. (2021), a Análise Documental utiliza procedimentos técnicos e científicos específicos para compreender o conteúdo dos documentos e deles obter informações relevantes conforme os objetivos da pesquisa.  

Serão analisados documentos oficiais, relatórios governamentais e bases estatísticas que tratam da atividade turística na Romênia, com ênfase na região de Bran. A coleta de dados quantitativos será realizada a partir de fontes oficiais, entre as quais se destacam o Instituto Nacional de Estatística da Romênia (INSSE), responsável pelas séries históricas sobre fluxo turístico e indicadores econômicos; a Prefeitura de Bran (Primăria Comunei Bran), que disponibiliza informações locais sobre receitas e gestão turística; a Organização Mundial do Turismo (OMT), que fornece parâmetros comparativos internacionais; e o Ministério da Economia e Turismo da Romênia, cujos relatórios oferecem dados sobre políticas públicas e desempenho do setor. Essas fontes permitirão examinar, de forma integrada, aspectos econômicos, institucionais e sociais que caracterizam o turismo literário no contexto estudado.


A análise qualitativa seguirá as orientações de Cechinel et al. (2016), que destacam a importância da crítica dos documentos quanto ao contexto, autoria, confiabilidade e relevância para o tema estudado. Dessa forma, cada documento será examinado de maneira criteriosa, considerando o contexto em que foi produzido, os autores responsáveis e sua credibilidade, a natureza e a finalidade do material, bem como os conceitos-chave e as informações que possam contribuir para sustentar a discussão teórica e empírica da pesquisa.

\section{Integração das abordagens qualitativa e quantitativa}

Com base em Minayo (2009) e Creswell (2007), a abordagem qualitativa busca compreender os significados e valores atribuídos ao fenômeno do turismo literário, enquanto a quantitativa contribui com dados objetivos e mensuráveis, permitindo identificar padrões e tendências econômicas. Essa integração metodológica conhecida como triangulação, é defendida por Tuzzo e Braga (2016) como forma de ampliar a validade e a profundidade interpretativa da pesquisa.  

A partir dessa combinação, o estudo analisará indicadores de fluxo turístico, ocupação hoteleira, sazonalidade e receita gerada, relacionando-os às representações simbólicas e culturais associadas à figura de \textit{Drácula}.  

\section{Fundamentação do método}

A adoção da Análise Documental como principal técnica qualitativa justifica-se pela sua capacidade de reunir e interpretar informações históricas, culturais e econômicas a partir de documentos diversificados, oferecendo uma visão contextualizada do objeto de estudo. Autores como Lüdke e André (1986), Cellard (2008) e Gil (2010) destacam que a análise documental é especialmente relevante quando o pesquisador busca compreender fenômenos sociais a partir de registros já existentes, o que se aplica ao presente estudo, dada a disponibilidade de dados oficiais e registros históricos sobre o turismo em Bran.  

Dessa forma, a metodologia proposta combina a robustez dos dados quantitativos oficiais com a profundidade interpretativa da abordagem qualitativa, permitindo avaliar o turismo literário de \textit{Drácula} como fenômeno cultural e econômico. Assim, busca-se compreender não apenas os números que sustentam a atividade turística, mas também os significados simbólicos e identitários que transformaram Bran em um dos destinos literários mais vampiresco da Europa.











\chapter{Estudo de Caso}
\section{Bran no Espelho}  
A identificação das Atividades Características do Turismo na comuna de Bran é fundamental para compreender como o turismo literário se articula com a economia local. A economia de Bran está fortemente vinculada à atividade turística. Nesse contexto, as ACTs, especialmente os serviços de hospedagem, alimentação, transporte e lazer, configuram-se como os principais pilares da sustentação econômica do município.

O transporte desempenha papel estratégico no sistema turístico de Bran. Por se tratar de uma localidade de pequeno porte, com infraestrutura urbana limitada, a comuna depende da acessibilidade proporcionada pelo condado de Brașov para garantir o fluxo constante de visitantes. As rotas rodoviárias que conectam Bran a Brașov e a Bucareste constituem o principal meio de acesso, permitindo que o destino seja integrado ao circuito turístico regional. Dessa forma, o transporte viabiliza o deslocamento dos turistas e funciona como indicador da vitalidade econômica do setor.

Segundo dados do Instituto Nacional de Estatística da Romênia (INSSE, 2025), o condado de Brașov consolidou-se, em 2023, como um dos principais destinos turísticos do país, ocupando a terceira posição nacional em número de turistas hospedados, atrás apenas de Bucareste e Constanța. O total de visitantes que pernoitaram na região cresceu 10,8\% em relação a 2022, enquanto o número de pernoites aumentou 12,1\%. Desde 2018, o número de estabelecimentos de hospedagem expandiu-se 30,4\%, posicionando Brașov em segundo lugar nacional em capacidade de acolhimento.

O turismo internacional também apresentou evolução expressiva, com aumento de 54,9\% no número de visitantes estrangeiros em comparação a 2022. Destacam-se os turistas provenientes da Alemanha, República da Moldávia, Israel e Estados Unidos. A taxa líquida de ocupação das estruturas de hospedagem passou de 25,9\% para 28,1\%, indicando maior eficiência na utilização da capacidade instalada e reforçando a atratividade turística regional. Esses dados evidenciam o diálogo das ACTs com a economia proveniente do turista.

Esta seção analisa a sazonalidade do turismo literário na região de Bran e seus impactos econômicos, considerando indicadores como fluxo de visitantes e ocupação hoteleira. A partir desses parâmetros, busca-se compreender as variações temporais da demanda turística e seus reflexos sobre a economia regional, destacando o papel das ACTs na estrutura produtiva local. Uma vez que Bran recebe a maior parte de seus turistas provenientes de Brașov, torna-se fácil mensurar o fluxo de visitantes que seguem essa rota.

\begin{table}[H]
\centering
\caption{Estruturas de alojamento turístico com função de hospedagem(2018–2023)}
\label{tab:estruturas_turisticas}
\begin{tabular}{lcccccc}
\hline
\textbf{Tipo de unidade de alojamento} & \textbf{2018} & \textbf{2019} & \textbf{2020} & \textbf{2021} & \textbf{2022} & \textbf{2023} \\ 
\hline
Unidades de alojamento – total & 961 & 930 & 902 & 1207 & 1229 & 1253 \\
Hotéis & 127 & 125 & 118 & 121 & 120 & 121 \\
Hostels & 26 & 26 & 25 & 25 & 23 & 19 \\
Apartamentos e quartos para alugar & – & – & – & 306 & 350 & 398 \\
Hotéis-apartamento & 2 & 1 & 3 & 3 & 3 & 2 \\
Motéis & 9 & 9 & 9 & 9 & 9 & 9 \\
Vilas turísticas & 84 & 82 & 77 & 69 & 64 & 57 \\
Chalés turísticos & 34 & 36 & 36 & 33 & 34 & 34 \\
Bangalôs & 5 & 5 & 5 & 6 & 6 & 11 \\
Vilas de férias & 2 & 2 & 2 & 1 & 1 & 2 \\
Campings & 2 & 2 & 2 & 2 & 6 & 6 \\
Casas turísticas & 3 & 3 & 3 & 4 & 2 & 4 \\
Acampamentos escolares & 1 & 1 & 1 & 1 & 1 & 1 \\
Pousadas turísticas & 264 & 255 & 242 & 240 & 230 & 217 \\
Pousadas rurais & 402 & 383 & 379 & 387 & 380 & 372 \\
\hline
\end{tabular}
\begin{flushleft}
\footnotesize \textbf{Fonte:} Instituto Nacional de Estatística da Romênia (2024). Dados adaptados do Capítulo 15 – Turismo, Condado de Brașov.
\end{flushleft}
\end{table}


A Tabela \ref{tab:estruturas_turisticas} demonstra a ampliação das estruturas de hospedagem entre 2018 e 2023, com destaque para o aumento expressivo de apartamentos e quartos para aluguel, que refletem o crescimento do turismo alternativo e da economia de compartilhamento na região. O incremento de unidades hoteleiras e pousadas rurais evidencia a diversificação da oferta de serviços e a adaptação do setor às novas demandas dos visitantes, reforçando o papel das ACTs como base do desenvolvimento turístico local.



\begin{table}[H]
\centering
\caption{Capacidade de alojamento turístico existente – 31 de julho (2018–2023)}
\label{tab:capacidade_existente}
\begin{tabular}{lcccccc}
\hline
\textbf{Tipo de unidade de alojamento} & \textbf{2018} & \textbf{2019} & \textbf{2020} & \textbf{2021} & \textbf{2022} & \textbf{2023} \\ 
\hline
Unidades de alojamento – total & 29\,832 & 29\,438 & 28\,726 & 33\,550 & 34\,169 & 34\,421 \\
Hotéis & 11\,928 & 11\,841 & 11\,369 & 11\,505 & 11\,648 & 11\,864 \\
Hostels & 965 & 885 & 954 & 1\,074 & 1\,003 & 732 \\
Apartamentos e quartos para alugar & – & – & – & 4\,663 & 5\,515 & 6\,201 \\
Hotéis-apartamento & 52 & 12 & 116 & 112 & 108 & 60 \\
Motéis & 630 & 594 & 630 & 630 & 630 & 630 \\
Vilas turísticas & 1\,853 & 1\,804 & 1\,787 & 1\,661 & 1\,497 & 1\,291 \\
Chalés turísticos & 1\,091 & 1\,161 & 1\,165 & 1\,113 & 1\,188 & 1\,152 \\
Bangalôs & 118 & 118 & 118 & 130 & 130 & 130 \\
Vilas de férias & 120 & 120 & 120 & 48 & 48 & 266 \\
Campings & 452 & 452 & 452 & 452 & 728 & 614 \\
Casas turísticas & 96 & 96 & 96 & 146 & 84 & 146 \\
Acampamentos escolares & 50 & 74 & 74 & 74 & 74 & 74 \\
Pousadas turísticas & 5\,583 & 5\,518 & 5\,150 & 5\,022 & 4\,828 & 4\,545 \\
Pousadas rurais & 6\,894 & 6\,763 & 6\,695 & 6\,920 & 6\,688 & 6\,716 \\
\hline
\end{tabular}
\begin{flushleft}
\footnotesize \textbf{Fonte:} Instituto Nacional de Estatística da Romênia (2024). Dados adaptados do Capítulo 15 – Turismo, Condado de Brașov.
\end{flushleft}
\end{table}

Os dados da Tabela \ref{tab:capacidade_existente} evidenciam o aumento contínuo da capacidade de alojamento, que passou de 29.832 unidades em 2018 para 34.421 em 2023, representando um crescimento de 15,4\%. Esse resultado reflete a expansão da infraestrutura turística e a consolidação de Brașov e Bran como destinos de alta atratividade. A diversificação dos tipos de hospedagem também revela uma tendência de descentralização e adaptação do turismo local às preferências contemporâneas dos visitantes.

\begin{table}[H]
\centering
\caption{Chegadas de visitantes estrangeiros no condado de Brașov, por país de origem (2018–2023)}
\label{tab:visitantes_estrangeiros}
\begin{tabular}{lcccccc}
\hline
\textbf{País / Região} & \textbf{2018} & \textbf{2019} & \textbf{2020} & \textbf{2021} & \textbf{2022} & \textbf{2023} \\
\hline
\textbf{Total geral} & 207\,488 & 191\,165 & 26\,685 & 54\,399 & 107\,595 & 166\,627 \\
\hline
\textbf{Europa – total} & 149\,487 & 137\,264 & 21\,990 & 44\,203 & 77\,716 & 124\,318 \\
União Europeia (total) & 126\,632 & 113\,254 & 14\,599 & 32\,770 & 52\,940 & 85\,397 \\
Alemanha & 25\,585 & 22\,926 & 3\,745 & 7\,563 & 13\,220 & 18\,845 \\
Itália & 11\,654 & 10\,286 & 1\,431 & 3\,287 & 5\,258 & 8\,600 \\
Espanha & 15\,821 & 15\,519 & 991 & 3\,143 & 4\,689 & 8\,543 \\
Polônia & 13\,304 & 8\,469 & 1\,193 & 4\,378 & 5\,116 & 8\,860 \\
França & 10\,277 & 9\,454 & 1\,753 & 3\,704 & 5\,168 & 7\,469 \\
Reino Unido & 13\,441 & 11\,449 & – & – & – & – \\
Bulgária & 5\,089 & 4\,480 & 709 & 1\,491 & 3\,602 & 7\,409 \\
Grécia & 3\,082 & 2\,696 & 598 & 631 & 1\,572 & 1\,956 \\
Bélgica & 2\,599 & 2\,774 & 393 & 1\,230 & 1\,555 & 3\,287 \\
Países Baixos & 4\,232 & 2\,716 & 497 & 1\,255 & 1\,948 & 4\,219 \\
Portugal & 1\,634 & 1\,400 & 84 & 264 & 483 & 748 \\
Outros países europeus & 10\,479 & 9\,163 & 1\,308 & 3\,941 & 6\,065 & 8\,972 \\
\hline
\textbf{Países não pertencentes à UE} & 22\,855 & 24\,010 & 7\,391 & 11\,433 & 24\,776 & 38\,921 \\
República da Moldávia & 7\,543 & 7\,168 & 2\,396 & 2\,774 & 7\,268 & 12\,676 \\
Reino Unido (não-UE pós-Brexit) & – & – & 2\,118 & 2\,833 & 7\,154 & 9\,858 \\
Suíça & 1\,862 & 1\,656 & 203 & 624 & 974 & 2\,614 \\
Turquia & 1\,716 & 1\,804 & 383 & 639 & 1\,342 & 2\,005 \\
Ucrânia & 3\,814 & 5\,266 & 923 & 1\,799 & 5\,450 & 7\,902 \\
Sérvia & 1\,074 & 896 & 202 & 200 & 715 & 1\,228 \\
\hline
\textbf{África} & 594 & 597 & 222 & 224 & 471 & 569 \\
Egito & 107 & 82 & 20 & 28 & 54 & 90 \\
Marrocos & 29 & 21 & 9 & 26 & 12 & 56 \\
Tunísia & 30 & 82 & 103 & 16 & 80 & 85 \\
\hline
\textbf{América do Norte} & 14\,417 & 13\,583 & 1\,135 & 3\,517 & 8\,390 & 13\,870 \\
EUA & 12\,475 & 11\,609 & 964 & 3\,168 & 7\,459 & 12\,215 \\
Canadá & 1\,942 & 1\,974 & 171 & 349 & 931 & 1\,655 \\
\hline
\textbf{América do Sul e Central} & 1\,736 & 1\,870 & 177 & 289 & 679 & 1\,337 \\
Brasil & 727 & 642 & 55 & 54 & 176 & 377 \\
México & 333 & 415 & 26 & 72 & 252 & 390 \\
Argentina & 203 & 357 & 17 & 29 & 68 & 108 \\
\hline
\textbf{Ásia} & 36\,204 & 35\,088 & 2\,934 & 5\,612 & 16\,425 & 23\,619 \\
Israel & 24\,895 & 23\,316 & 1\,993 & 4\,499 & 14\,095 & 17\,358 \\
China & 4\,700 & 5\,700 & 240 & 161 & 375 & 1\,788 \\
Japão & 2\,658 & 1\,988 & 235 & 52 & 173 & 466 \\
Índia & 595 & 701 & 69 & 79 & 291 & 466 \\
Irã & 121 & 304 & 28 & 42 & 122 & 549 \\
\hline
\textbf{Oceania} & 2\,002 & 2\,257 & 135 & 169 & 553 & 1\,721 \\
Austrália & 1\,567 & 1\,835 & 92 & 88 & 427 & 1\,429 \\
Nova Zelândia & 305 & 331 & 23 & 40 & 61 & 137 \\
\hline
\textbf{Não especificados} & 3\,048 & 506 & 92 & 385 & 3\,361 & 1\,193 \\
\hline
\end{tabular}
\begin{flushleft}
\footnotesize \textbf{Fonte:} Instituto Nacional de Estatística da Romênia (2024). Dados adaptados do Capítulo 15 – \textit{Turismo}, Condado de Brașov.
\end{flushleft}
\end{table}
A Tabela \ref{tab:visitantes_estrangeiros} mostra o crescimento expressivo do turismo internacional no condado de Brașov, com destaque para o retorno gradual após a pandemia de COVID-19. O aumento significativo de visitantes estrangeiros, sobretudo da Europa Ocidental e da América do Norte, evidencia o fortalecimento da imagem da Transilvânia como destino literário e cultural. A presença de turistas provenientes de países como Alemanha, Itália e Espanha reforça a relevância do turismo literário vinculado ao mito de \textit{Drácula} no contexto europeu contemporâneo.



\begin{figure}[h!]
    \centering
    \caption{Duração média do tempo de estadia em Brașov, dados de 2025.}
    \includegraphics[width=1\linewidth]{tempo-estadia.png}
    
    \label{fig:durata_sejur}
    \vspace{2mm}
    {\footnotesize Fonte: Instituto Nacional de Estatística da Romênia – Brașov (2025). Disponível em: \url{https://brasov.insse.ro/comunicate-de-presa/turism/turism/#respond}. Acesso em: 28 out. 2025.}
\end{figure}

A figura apresentada ilustra o tempo médio de permanência no Condado de Brașov ao longo do período de janeiro de 2022 até julho de 2025, discriminado entre romenos (Romani), estrangeiros(Straini) e o total de turistas. No gráfico, observa-se que os turistas estrangeiros costumam permanecer mais tempo no condado, com valores próximos ou acima de 2,5 dias em diversos períodos, enquanto os turistas romenos apresentam um tempo médio de permanência mais estável, geralmente em torno de 2 dias. A linha referente ao total de turistas se situa entre essas duas curvas, refletindo a média geral.

\section{Brașov e Bran o lugur onde o Sangue vira Ouro.}

O investimento, em seu sentido mais amplo, é um motor essencial para o desenvolvimento e crescimento econômico de uma região. Embora o investimento direto nas \textit{Atividades Características do Turismo (ACTs)} seja crucial, o investimento em outros setores da economia desempenha um papel igualmente fundamental, gerando um \textbf{impacto indireto} que molda a experiência do turista e a competitividade do destino.

A robustez econômica geral, impulsionada por investimentos em áreas como a \textbf{Indústria de Transformação}, \textbf{Comércio} ou \textbf{Administração Pública}, cria um ambiente de negócios mais estável e próspero. Esse dinamismo se traduz em melhorias na infraestrutura básica, maior disponibilidade de serviços de apoio e um aumento na renda e qualidade de vida da população local, fatores que, em última análise, beneficiam o setor de turismo.

\subsection{O Impacto dos Investimentos em Brașov e como isso afeta os Turista?}

O investimento em setores não diretamente ligados ao turismo influencia as ACTs de diversas maneiras. Por exemplo, o investimento em \textbf{Transporte e Armazenamento} (que pode incluir a modernização de portos, aeroportos e malhas rodoviárias) facilita o acesso ao destino, reduzindo custos e tempo de viagem para o turista. Da mesma forma, o investimento em \textbf{Distribuição de Água e Saneamento} e \textbf{Saúde e Assistência Social} eleva o padrão de vida e a segurança sanitária da região, tornando-a mais atrativa e confiável para visitantes internacionais e domésticos.

A diversificação da economia, apoiada por investimentos em diferentes domínios de atividade, gera uma maior oferta de produtos e serviços de apoio que o turista pode consumir, enriquecendo a \textit{experiência turística} para além das atividades tradicionais. Quando a economia local é forte e diversificada, os serviços de apoio (como bancos, telecomunicações e comércio varejista) são mais eficientes, o que se reflete na satisfação do turista.

\subsection{A Importância da Distribuição do Investimento}

A análise da distribuição dos investimentos líquidos por domínio de atividade (como ilustrado na Tabela 1) permite identificar a estrutura econômica de uma região e inferir seu potencial de impacto indireto no turismo. Um país ou região que concentra a maior parte de seus investimentos em setores de alta produtividade e tecnologia, mesmo que não sejam as ACTs, está, indiretamente, fortalecendo a base para um turismo mais sofisticado e sustentável.

Embora o investimento direto em \textbf{Hotéis e Restaurantes} (uma ACT primária) seja vital para a qualidade da oferta imediata, a sustentabilidade e a competitividade de longo prazo do destino dependem da capacidade da economia local de absorver e gerar valor em múltiplos setores. O investimento em \textbf{Construção} e \textbf{Transações Imobiliárias}, por exemplo, não apenas cria novas infraestruturas para o turismo, mas também melhora a qualidade da paisagem urbana e rural, que são parte integrante do produto turístico.

Portanto, a relação entre investimento e as \textit{Atividades Características do Turismo} não é apenas direta, mas profundamente interligada ao \textbf{desenvolvimento econômico geral}. O investimento em setores não turísticos é um pré-requisito para criar a infraestrutura, a estabilidade e a diversidade de serviços que elevam a qualidade e a competitividade do destino turístico como um todo.

\begin{table}[H]
\centering
\caption{Estrutura dos Investimentos Líquidos por Domínio de Atividade — Condado de Brașov (Semestre I de 2025)}
\label{tab:investimentos_brasov}
\begin{tabular}{lcc}
\hline
\textbf{Domínio de Atividade} & \textbf{2025} & \textbf{2024} \\ \hline
Indústria de Transformação & 56,8\% & 53,9\% \\
Administração Pública e Defesa & 17,3\% & 17,1\% \\
Comércio (Atacado e Varejo) & 7,4\% & 11,1\% \\
Saúde e Assistência Social & 4,0\% & 2,8\% \\
Distribuição de Água e Saneamento & 3,5\% & 2,2\% \\
Construção & 3,9\% & 4,6\% \\
Transporte e Armazenamento & 2,4\% & 2,3\% \\
Hotéis e Restaurantes & 1,2\% & 0,7\% \\
Transações Imobiliárias & 1,9\% & 0,2\% \\
Agricultura, Silvicultura e Pesca & 0,4\% & 1,2\% \\
Outros & 1,2\% & 1,9\% \\ \hline
\textbf{Total} & \textbf{100,0\%} & \textbf{100,0\%} \\ \hline
\end{tabular}

\vspace{2mm}
{\footnotesize Fonte: Direcţia Judeţeană de Statistică Braşov (DJS Braşov), “Investiţii ‒ Judeţul Braşov, 2025”. Tradução nossa. - Adaptado}
\end{table}


\begin{table}[H]
\centering
\caption{Índices de Volume e Estrutura dos Investimentos Líquidos — Semestre I de 2025 em comparação com o Semestre I de 2024}
\label{tab:indici_de_volum}
\resizebox{\textwidth}{!}{%
\begin{tabular}{lccc}
\hline
\textbf{Categoria de Investimento} & \textbf{Volume (\%)} & \textbf{2024, (\%)} & \textbf{ 2025, (\%)} \\ \hline
\textbf{Total} & 101,0 & 100,0 & 100,0 \\
Obras de Construção Novas & 104,0 & 59,9 & 62,2 \\
Equipamentos* & 94,0 & 27,9 & 25,1 \\
Outras Despesas & 102,1 & 12,2 & 12,7 \\ \hline
\end{tabular}%
}
\vspace{2mm}
{\footnotesize Fonte: Direcţia Judeţeană de Statistică Braşov (DJS Braşov), “Investiţii ‒ Judeţul Braşov 2025”. Tradução nossa.}
\end{table}



Essa estrutura de investimento, que prioriza a construção e a estabilidade econômica geral, coincide com um notável crescimento na demanda turística. No mesmo período, o Condado de Brașov consolidou sua posição, registrando um aumento expressivo de 21,6\% no número de chegadas de turistas em janeiro de 2025 em comparação com o ano anterior. Esse crescimento, que coloca Brașov no topo do ranking nacional de chegadas (15,6\% do total nacional), sugere que a melhoria da infraestrutura geral e a percepção de um destino economicamente sólido e bem administrado, mesmo que os investimentos diretos em ACTs sejam modestos, contribuem decisivamente para a atração de visitantes.

Em suma, a correlação indica que o crescimento do turismo não é apenas resultado de investimentos diretos e específicos, mas também um reflexo da saúde econômica ampla da região. A ênfase em obras de construção e a manutenção de investimentos em setores não turísticos criam o ambiente de apoio e a infraestrutura básica que tornam o destino mais acessível, seguro e atraente, impactando positivamente as Atividades Características do Turismo.


\section{Bran no Espelho de Outros Destinos Literários}





% ----------------------------------------------------------
% resultados
% ----------------------------------------------------------
\chapter{Considerações Finais}







% ----------------------------------------------------------
% REFERÊNCIAS
% ----------------------------------------------------------
\postextual
\renewcommand{\refname}{Referências}

\begin{thebibliography}{99}
\bibitem{butler2012} BUTLER, R. \textbf{The concept of a tourist area cycle of evolution: implications for management of resources}. Canadian Geographer, 2012. \url{https://www.researchgate.net/publication/228003384_The_Concept_of_A_Tourist_Area_Cycle_of_Evolution_Implications_for_Management_of_Resources}

\bibitem{frost2016} FROST, W. \textbf{Turismo e Balcanismo, a partir do Dracula de Bram Stoker}, v. 57, p. 45–58, 2016. \url{https://www.academia.edu/27446060/Turismo_e_balcanismo_a_partir_do_Drácula_de_Bram_Stoker}

\bibitem{baleiro2018} BALEIRO, R.; QUINTEIRO, S. \textbf{Literary Tourism: An Interdisciplinary Approach}. Channel View Publications, 2018. \url{https://www.channelviewpublications.com/display.asp?k=9781845416726}

\bibitem{dwyer2004} DWYER, L.; FORSYTH, P.; SPURR, R. \textbf{Evaluating tourism’s economic effects: new and old approaches}. Tourism Management, v. 25, n. 3, p. 307–317, 2004. \url{https://www.sciencedirect.com/science/article/pii/S026151770300066X}

\bibitem{fletcher2018} FLETCHER, J. et al. \textbf{Tourism: Principles and Practice}. Pearson, 2018. \url{https://www.sciencedirect.com/science/article/abs/pii/S0261517703001316}

\bibitem{reijnders2011} REIJNDERS, S. (2011). \textit{Stalking the Count: Dracula, Fandom \& Tourism}. \textit{Annals of Tourism Research}, 38(1), 231--248. \url{https://www.researchgate.net/publication/241860994_Stalking_the_count_Dracula_Fandom_and_Tourism}

\bibitem{ibge2003}
BRASIL. Ministério do Planejamento, Orçamento e Gestão; Instituto Brasileiro de Geografia e Estatística (IBGE). \textit{Economia do turismo: análise das atividades características do turismo – 2003}. Rio de Janeiro: IBGE, 2006. 62 p. (Estudos e Pesquisa. Informação Econômica, n. 5). ISBN 85-240-3923-X. Disponível em: \url{https://www.gov.br/turismo/pt-br/acesso-a-informacao/acoes-e-programas/observatorio/repositorio/economia-do-turismo/economia_turismo___dados_de_2003.pdf}. Acesso em: 27 out. 2025.

\bibitem{ibge2003}
BRASIL. Ministério do Planejamento, Orçamento e Gestão; Instituto Brasileiro de Geografia e Estatística (IBGE). \textit{Economia do turismo: análise das atividades características do turismo – 2003}. Rio de Janeiro: IBGE, 2006. 62 p. (Estudos e Pesquisa. Informação Econômica, n. 5). ISBN 85-240-3923-X. Disponível em: \url{https://www.gov.br/turismo/pt-br/acesso-a-informacao/acoes-e-programas/observatorio/repositorio/economia-do-turismo/economia_turismo___dados_de_2003.pdf}. Acesso em: 27 out. 2025.

\bibitem{smith1996}
SMITH, Adam. \textit{A riqueza das nações: investigação sobre sua natureza e suas causas}. São Paulo: Nova Cultural, 1996.

\bibitem{ribeiro2021}
RIBEIRO, Luiz Carlos de Santana; SANTOS, Monique Manuela Carvalho dos; SANTOS, Fernanda Rodrigues dos. \textit{Avaliação das Atividades Características do Turismo no Brasil: 2012–2020}. \textit{Turismo: Visão e Ação}, v. 23, n. 3, p. 557–578, 2021. DOI: \url{https://doi.org/10.14210/rtva.v23n3.p557-578}.

\bibitem{lima2021}
LIMA JUNIOR, José Alves; OLIVEIRA, Rita de Cássia; SANTOS, Marisa; SCHNEKENBERG, Marilene. \textit{Análise Documental como percurso metodológico na pesquisa qualitativa}. Revista FUCAMP, v. 24, n. 47, p. 12–25, 2021. Disponível em: \url{https://revista.fucamp.edu.br/index.php/revistafucamp/article/view/2742}.  

\bibitem{sa2009}
SÁ-SILVA, Jackson Ronie; ALMEIDA, Cristóvão Domingos de; GUINDANI, Joel Felipe. \textit{Pesquisa documental: pistas teóricas e metodológicas}. Revista Brasileira de História \& Ciências Sociais, v. 1, n. 1, p. 1–15, 2009.  

\bibitem{cellard2008}
CELLARD, André. \textit{A análise documental}. In: POUPART, Jean et al. (org.). \textit{A pesquisa qualitativa: enfoques epistemológicos e metodológicos}. Petrópolis: Vozes, 2008. p. 295–316.  

\bibitem{ludke1986}
LÜDKE, Menga; ANDRÉ, Marli Eliza Dalmazo Afonso de. \textit{Pesquisa em educação: abordagens qualitativas}. São Paulo: EPU, 1986.  

\bibitem{minayo2009}
MINAYO, Maria Cecília de Souza. \textit{O desafio do conhecimento: pesquisa qualitativa em saúde}. 12. ed. São Paulo: Hucitec, 2009.  

\bibitem{tuzzo2016}
TUZZO, Simone Amado; BRAGA, Rosane de Fátima. \textit{Triangulação e pesquisa em comunicação}. Revista FAMECOS, v. 23, n. 3, p. 1–15, 2016. DOI: \url{https://doi.org/10.15448/1980-3729.2016.3.23314}.  

\bibitem{cechinel2016}
CECHINEL, Anderson; ALBINO, Jéssica; SEBASTIÃO, Liliane. \textit{Análise documental e pesquisa qualitativa: uma aproximação metodológica}. Anais do Congresso Nacional de Educação, Curitiba, 2016.  

\bibitem{gil2010}
GIL, Antonio Carlos. \textit{Como elaborar projetos de pesquisa}. 5. ed. São Paulo: Atlas, 2010.  

\bibitem{brasov2025}
DIRECȚIA JUDEȚEANĂ DE STATISTICĂ BRAȘOV. \textit{Investiții}. Disponível em: \url{https://brasov.insse.ro/comunicate-de-presa/investitii/investitii/#respond}. Acesso em: 28 out. 2025.


\bibitem{constantino2019}
CONSTANTINO, A. \textit{Fluxos turísticos entre os países do Corredor Bioceânico}. \textit{Interações}, v. 20, n. 2, p. 123-145, 2019. Disponível em: \url{https://www.scielo.br/j/inter/a/JZpXf4RfYwrpmJWCWpVLHCB/?lang=pt}. Acesso em: 28 out. 2025.


\end{thebibliography}



\end{document}
