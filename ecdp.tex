% ----------------------------------------------------------
% Modelo ABNT completo — Trabalho de Pesquisa
% ----------------------------------------------------------
\documentclass[
12pt,                
oneside,             
a4paper,             
chapter=TITLE,       
english,             
brazil               
]{abntex2}

% ----------------------------------------------------------
% PACOTES BÁSICOS
% ----------------------------------------------------------
\usepackage[utf8]{inputenc}      
\usepackage[T1]{fontenc}         
\usepackage{lmodern}             
\usepackage{indentfirst}         
\usepackage{graphicx}            
\usepackage{microtype}           
\usepackage{booktabs}            
\usepackage{array}
\usepackage{setspace}
\usepackage{hyperref}
\usepackage{float}
\usepackage{lipsum}
\usepackage{pgfplots}
\pgfplotsset{compat=1.18}
\usepackage{amsmath} 
\usepackage{hyperref}

% ----------------------------------------------------------
% CONFIGURAÇÕES DE DOCUMENTO
% ----------------------------------------------------------
\setlength{\parindent}{1.25cm}
\setlength{\parskip}{0.2cm}
\OnehalfSpacing

\titulo{IMPACTOS ECONÔMICOS DO TURISMO LITERÁRIO DE “DRÁCULA” NO VILAREJO DE BRAN, NA ROMÊNIA}
\autor{Júlia Michetti Costa Santos\\ Fernanda França Silva}
\local{Belo Horizonte}
\data{2025}
\tipotrabalho{Trabalho de Pesquisa — Estágio Curricular}
\preambulo{%
Trabalho apresentado à disciplina \textit{Estágio Curricular de Pesquisa} do curso de graduação em Turismo da Universidade Federal de Minas Gerais.}

% ----------------------------------------------------------
% REMOVE ORIENTADOR E INSTITUIÇÃO DA FOLHA DE ROSTO
% ----------------------------------------------------------
\makeatletter
\renewcommand{\imprimirorientador}{}
\renewcommand{\imprimirinstituicao}{}
\makeatother

% ----------------------------------------------------------
% INÍCIO DO DOCUMENTO
% ----------------------------------------------------------
\begin{document}
\selectlanguage{brazil}
\frenchspacing

% CAPA
\imprimircapa

% FOLHA DE ROSTO
\imprimirfolhaderosto*

% ----------------------------------------------------------
% ELEMENTOS PRÉ-TEXTUAIS
% ----------------------------------------------------------

\begin{resumo}
O presente trabalho busca compreender os impactos econômicos do turismo literário relacionado à obra “Drácula”, de Bram Stoker, no vilarejo de Bran, na Romênia. O estudo pretende examinar como o mito do vampiro influencia o fluxo turístico, a geração de renda e o desenvolvimento local, a partir de uma abordagem interdisciplinar entre turismo, literatura e economia. 

\vspace{\onelineskip}
\noindent
\textbf{Palavras-chave}: turismo literário; economia do turismo; Drácula; Romênia; Bran.
\end{resumo}



% ----------------------------------------------------------
% Sumario
% ----------------------------------------------------------
\tableofcontents % SUMÁRIO automático

% ----------------------------------------------------------
% ELEMENTOS TEXTUAIS
% ----------------------------------------------------------
\textual

\newpage
\chapter{Introdução}
De acordo com Butler (2012), o turismo literário se caracteriza pela motivação de visitar lugares que mantêm alguma relação com a literatura, sejam espaços que inspiraram narrativas, residências de autores ou locais associados a personagens e enredos ficcionais. Nessa perspectiva, a literatura ultrapassa seu papel artístico e passa a influenciar a materialidade dos lugares, transformando o imaginário em destino turístico. 

O Castelo de Bran, na Romênia, representa um exemplo emblemático desse fenômeno: Na história, o personagem Jonathan Harker, um jovem advogado inglês, chega à Transilvânia para tratar de negócios, mas logo se vê preso em um castelo envolto em mistérios. Atordoado pelo ambiente, Harker torna-se refém de sua própria trajetória, registrando em cartas e documentos a experiência que atravessa o limiar entre o real e o imaginário. De forma análoga, os visitantes contemporâneos percorrem o castelo de Bran, movidos pela curiosidade e pela fantasia criada por Bram Stoker, buscando reviver, a aventura do personagem. Como observa Stijn Reijnders em Stalking the Count: Dracula, Fandom e Tourism (2011), esses turistas estabelecem uma conexão emocional e imaginativa com os locais associados à obra, transformando a visita em uma experiência estética e narrativa.


Como observa Frost (2016, p. 45), “o vampiro criado por Bram Stoker e o mito de Drácula influenciaram diretamente a maneira como o governo romeno passou a trabalhar o turismo, principalmente para demandas ocidentais”. Nessa mesma linha, pesquisas em literatura e turismo, como as de Baleiro e Quinteiro (2018), evidenciam que a relação entre texto literário e território ultrapassa o campo simbólico e configura-se como uma área de estudo interdisciplinar, na qual se entrelaçam cultura, memória e economia.

A importância de mensurar esses efeitos na economia é destacada por Dwyer, Forsyth e Spurr (2004, p. 308), ao afirmarem que “a análise dos impactos econômicos do turismo oferece a evidência mais tangível e convincente da contribuição do setor para o desenvolvimento regional”. 

Essa perspectiva reforça a necessidade de observar o fenômeno do turismo literário não apenas em sua dimensão simbólica, mas também como motor de crescimento. No caso do vilarejo de Bran, o mito de Drácula consolidou o castelo como principal atrativo da região e como um dos destinos mais visitados da Romênia, influenciando diretamente o desenvolvimento econômico local, e que de acordo com Fletcher et al. (2018) ressalta que a mensuração dos impactos financeiros do turismo é um dos instrumentos mais concretos para compreender a evolução de um destino, pois reflete diretamente sua capacidade de gerar receita, empregos e investimentos locais fator que ainda é pouco estudado embora seja o meio mais tangível de medir a evolução de turismo de um local.

Esse estudo se propoe a compreender a forma que o turismo literário associado à obra Drácula impacta a economia do vilarejo de Bran, na Romênia, é fundamental para analisar as transformações econômicas decorrentes da atividade turística na região.












% ----------------------------------------------------------
% Referencial teorico
% ----------------------------------------------------------
\chapter{Referencial Teorico}





%turismo literario
\section{Entre páginas: O despertar turistico de Drácula}

Antes de compreender o conceito de turismo literário, é necessário analisar a definição de atividade turística. Segundo a Organização Mundial do Turismo (OMT), o turismo é uma prática que está subordinada à viagem, sendo o viajante uma figura que se desloca entre pontos geográficos diferentes, motivado por diversos objetivos específicos e com durações definidas. No glossário da OMT, o turismo é descrito como:

\begin{quote}
"Tourism is a social, cultural and economic phenomenon which entails the movement of people to countries or places outside their usual environment for personal or business/professional purposes. These people are called visitors (which may be either tourists or excursionists; residents or non-residents) and tourism has to do with their activities, some of which involve tourism expenditure." \cite{UNTourismIndicators}
\end{quote}

Em tradução livre, entende-se que o turismo é uma atividade social, cultural e um fenômeno econômico que implica o movimento de pessoas para países ou locais fora de seu ambiente usual, com propósitos pessoais, de negócios ou profissionais. Essas pessoas são chamadas visitantes (que podem ser turistas ou excursionistas, residentes ou não residentes) e o turismo está relacionado com suas atividades, algumas das quais envolvem gastos turísticos.

O turista é um tipo particular de viajante, que se diferencia, de acordo com Quinteiro e Baleiro (2014), por ser um indivíduo em busca do simulacro (ou seja, situações que encenam a experiência do que pode ou não ser autêntico) e que não possui nenhuma sabedoria a transmitir. Dentro da atividade turística, temos variadas ramificações das possibilidades e motivações para os quais os indivíduos se deslocam. O turismo cultural, definido no glossário da OMT por:

\begin{quote}
"É um tipo de atividade turística em que a motivação essencial dos visitantes é de aprender, descobrir, experienciar ou consumir atrações ou produtos tangíveis e intangíveis de um destino turístico. Essas atrações ou produtos podem ser relacionados a uma diversidade de características materiais, intelectuais, espirituais e emocionais de uma sociedade, que engloba arte e arquitetura, heranças culturais e históricas, herança culinária, literatura, música, indústrias criativas e suas culturas vivas, assim como seus estilos de vida, sistemas de valor, crenças e tradições." \cite{UNTourismIndicators}
\end{quote}

Dentro do turismo cultural há uma subdivisão do turismo literário que parte também da literatura de turismo. A literatura de turismo traz em ênfase a compreensão do que é turismo literário, e sua relação direta com o lugar literário. Para Quinteiro e Baleiro, a literatura de turismo é definida como uma relação entre o mundo da imaginação e o mundo físico que se concretiza, por meio de uma deslocação do turista ao lugar literário. E, continuando a definir, o lugar literário é aquele que possui particularidades de sua construção em que são medidas pela literatura, transferindo para a paisagem física. Esses lugares podem ser representados em textos literários ou por inspirações dos mesmos, ou seja, para o primeiro, os lugares são determinados pela motivação pelo desejo de encontrar na paisagem real o que foi descrito em um livro; a busca pelo produto da imaginação. Já para o segundo, determina uma associação à figura do autor.

O turismo literário tem uma ampla e variada definição que se complementa por diversos autores encontrados, e que determinam de forma exemplar a importância desse termo para determinados territórios. Para Graham Busby (2022), o termo se define de registros mais antigos, de mais de 200 anos atrás, demonstrado pelo autor que um dos exemplos mais claros para tal seria a literatura de Shakespeare. Para Busby, o turismo literário engloba visitas relacionadas a moradia dos escritores, lápides dos mesmos, locais identificáveis de livros e aqueles que são acreditados de estar em livros, podendo ocorrer em áreas urbanas ou rurais, sendo atividades geralmente realizadas por visitantes considerados de boa carga intelectual.

Quinteiro e Baleiro (2014) descrevem o turismo literário com similaridade à Busby. De acordo com esses, a investigação multidisciplinar em turismo e literatura é muito recente e traz para o campo termos importantes para a compreensão do tipo de turismo determinado como literário, que não serão abordados aqui. Mas podemos evidenciar a importância de saber a diferença entre peregrino literário e o turista literário. O primeiro se define por ser aquele que, movido por uma profunda admiração por um autor, viaja longas distâncias com objetivo de experimentar o que o autor admirou, o que ele viu, o que ele sentiu, estar onde viveu, onde morreu, todos os fatores que podem estar em conexão real com o autor. Assim como o turista literário é aquele que tem vontade de procurar, identificar e confirmar os conhecimentos proporcionados pelo autor, contribuindo para própria construção do lugar literário (Quinteiro e Baleiro, 2014).

Pode-se apontar que há uma concordância entre os autores quanto à ampliação e pré-existência de um capital cultural e simbólico, onde o consumo da literatura é um diferenciador de classes.

Samet Çevik (2020) demonstra que o desenvolvimento do turismo literário causou transformações de locais literários para contribuir para o desenvolvimento do turismo e estimular a economia local, e pode beneficiar esses locais financeiramente a partir da receita gerada por atividades correlatas à atividade turística. Ainda de acordo com o autor, turismo literário é um tipo de turismo que tem relação próxima com outros tipos de turismo, e tem várias ramificações dentro de si. Podemos citar como tipos de turismo correlatos, o turismo cinematográfico e o dark turismo.

\begin{quote}
"O turismo cinematográfico tem como principal motivação a visitação de locais relacionados com produções audiovisuais." - Melo et al. (2024)
\end{quote}

Há uma relação dos produtos turísticos com a localidade que foi utilizada como cenários das filmagens de produções audiovisuais. Essa prática turística tem motivação relacionada ao interesse por produções audiovisuais e por equipamentos ou eventos relacionados, ainda de acordo com Melo et al. (2024). E é uma experiência que está abordada em locais que foram retratados, de alguma forma, em algum meio de comunicação.

Outrossim, o dark turismo é definido por Pereira et al. (2022) por envolver atrações que estão associadas a desastres naturais, ou influenciados pelo homem, ou atrocidades que transformaram os locais em mais do que lugares de memória, mas também atrações turísticas. O dark turismo é considerado para esse autor, um tipo contemporâneo e relevante, que tem abordagem investigacional do relacionamento entre o turismo e a morte. As pessoas são atraídas, propositalmente ou não, por diversos instrumentos turísticos (sites, atrações, eventos) que têm relação com a morte, sofrimento, violência ou desastres. Apesar de ter um crescimento considerável de locais que são considerados como dark turismo, a motivação dos indivíduos para a prática dessa atividade ainda é uma incógnita.



%o vilarejo de bran
\section{Bran: O coração sombrio da Romênia}

O vilarejo de Bran, local de estudo deste trabalho, está localizado no território do condado da Valáquia, na fronteira com o condado da Transilvânia, na Romênia. O local encontra-se no distrito de Brasov, com uma área de 67,85 km² e uma população de 5866 habitantes, segundo o censo de 2021 \cite{brasov2021}. Nesse local se localiza o castelo de Bran, que possui site próprio, descrito nas referências, e é um grande atrativo do turismo literário \cite{castelobran}.

\begin{figure}[H]
    \centering
    \includegraphics[width=1\linewidth]{mapa-1.png}
    \caption{Localizações Associadas aos Dráculas Fictícios e Históricos. A imagem ilustra as regiões ligadas à figura histórica de Vlad III, o Empalador, e à representação fictícia de Drácula, inspirada no monarca medieval romeno. Fonte: J. R. Hall, Dracula tourism in Romania: Cultural identity and the state. Disponível em: \url{https://doi.org/10.1016/j.annals.2007.03.004} Acesso em: 21 nov. 2025.}
    \label{fig:placeholder}
\end{figure}
 
O castelo de Bran, localizado no vilarejo de Bran, em Brasov, é um exemplo claro de atrativo turístico que engloba o turismo literário, o turismo cinematográfico e o dark turismo, por relacionar a obra de Bram Stoker, o filme “O Drácula” e também a relação com o personagem correlacionado ao qual inspirou o vampiro fictício, Vlad III (popularmente conhecido como “O Impalador”). 

Sarvenaz Safavi (2025) descreve a história de Vlad III fazendo um estudo comparativo de sua imagem com a imagem do Conde Drácula, inventado por Bram Stoker. Em seu artigo, Safavi relata os inúmeros conflitos civis que ocorreram no território da Romênia – até sua unificação e reconhecimento de independência – e a importância para a sociedade romena da imagem de Vlad III para essa conquista, por sua peculiar forma de torturar e reivindicar seu poder.

Outrossim, a casa de Anne Frank é citada por Çevik (2020) como um outro exemplo claro de destino turístico com motivação dark, e pode-se relacionar o estabelecimento também como um ambiente com proposta de reconhecimento pelo turismo literário, visto que também possui uma obra literária diretamente conectada a um local físico.

Destinos como o castelo de Bran podem promover atividades que podem beneficiar economicamente, visto que, de acordo com Çevik (2020), os destinos que promovem atividades de turismo literário como uma atração podem dignificar a receita gerada dessas atividades, a depender do aumento do número de turistas. Para a Romênia, a ocorrência de turistas no território é um fator importante e influente, pois, de acordo com Turnock (1977):

\begin{quote}
“A provisão para o turismo doméstico e internacional está assumindo um crescimento importante em países socialistas.”
\end{quote}

Naquele cenário, havia diversos fatores de ocorrência de turistas que eram interpretadas como desafiantes para os governos socialistas, e a Romênia – tendo sua economia inicialmente mineradora, começou a ter a necessidade de atender demandas de hospedagem com motivações de excursões advindas de instituições de ensino e grupos sindicalizados. O autor demonstra que o turismo, nos primeiros anos comunistas do país, teve pouca prioridade e quase não é mencionado em discursos políticos. Posteriormente, o turismo começou a ter importância a partir dos estudos e planejamentos geográficos, e conectando com o texto de Duncan Light (2007), o governo romeno não tinha ciência da existência da obra de Stoker e do filme “O Drácula”, considerando que a atividade turística é diretamente relacionada pelos representantes à geografia e ao território, de acordo com Turnock. Este também expõe que o Conde Drácula é uma atração especial que influencia visitas inspiradoras à história local e pode levar visitantes a áreas remotas que proporcionam outros tipos de turismo, acreditando que as forças que agem sobre o "efeito expansionista" na economia socialista podem não ser substancialmente diferentes do sistema capitalista, mas a experiência romena com o turismo sugere que o tempo de decisão para compromissar com o instinto de realizar investimentos pode ser menor do que o esforço em contextos de regiões com melhores instrumentos de planejamento.

\begin{figure}[H]
    \centering
    \includegraphics[width=1\linewidth]{vlad.png}
    \caption{Retrato de Vlad III, o Empalador, ou Drácula (1431-1476), do século 16. A imagem é uma representação do monarca medieval romeno, cuja figura histórica inspirou o personagem fictício de Bram Stoker.\\[0.3cm]  \textbf{Fonte:} BBC. Retrato de Vlad III, o Empalador. Disponível em: \url{https://www.bbc.com/portuguese/articles/c51pyzpezd8o}. Acesso em: 21 nov. 2025.}
    \label{fig:vlad}
   
\end{figure}



Duncan (2007) também diz que o turismo de Drácula é problemático por ser gerado a partir de uma demanda externa, reforçando imagens errôneas sobre a Romênia e sua história. A obra de Bram Stoker foi escrita, de acordo com o autor supracitado, sem que o mesmo tenha visitado o país, criando uma representação mística e sinistra e desvalorizando a história e a imagem de Vlad III, “O Impalador”. Os visitantes que iam inicialmente ao castelo, como explicado no artigo de Duncan (2007), não tinham conhecimento sobre a história local e acabavam por visitar locais incorretos, desencadeando em uma adoção informal do castelo de Bran como o “castelo original” da narrativa de Stoker.

A partir disso, o governo romeno teve que tomar iniciativas sobre a sua atuação no mercado do turismo, pois suas atividades características do turismo (atividades que tipicamente produzem produtos característicos do turismo) e seus produtos característicos do turismo (os que satisfazem um ou ambos desses critérios: os produtos que deveriam representar um compartilhamento significativo das despesas totais do turismo e os produtos que representam significativa consideração dos fornecimentos de produtos na economia) \footnote{Ambas definidas pelo glossário oficial da Organização Mundial do Turismo} eram definidas prioritariamente por forças externas.

%Impactos econômicos e indicadores
\section{Economia: O Conde, Uma Pequena Introdução}

No clássico Drácula, escrito por Bram Stoker, a história se desenrola ao redor do enigmático conde, que vive em um castelo isolado na Transilvânia, longe das luzes da civilização. O conde Drácula, uma criatura noturna e sedenta por sangue, busca expandir seu poder além dos limites de sua terra natal.

A relação entre a história e a economia é intrínseca ao próprio simbolismo da figura de Drácula. Sua busca por mais terras e sua tentativa de estabelecer controle sobre novas regiões podem ser vistas como uma metáfora para o expansionismo econômico, onde a busca por novos mercados e recursos impulsiona o crescimento de um império. Fora da fantasia, essa história se transforma em um motor econômico para a região da Transilvânia, com o turismo atraído pela lenda do conde, tornando-se uma fonte de renda e desenvolvimento. 

A jornada para compreender a economia do turismo literário de é então um delicioso banquete, antes mesmo de imergimos em dados é necessário remontar às origens do pensamento econômico moderno. Adam Smith (1776) concebeu a economia como o estudo da produção, distribuição e consumo da riqueza das nações, identificando o trabalho humano como a força vital dessa riqueza. Sua visão inaugural permanece relevante, servindo como uma base conceitual para a análise contemporânea:

\begin{quote}
“O trabalho anual de cada nação é o fundo que originalmente a supre com todas as necessidades e conveniências da vida que anualmente consome, e que consistem sempre, ou no produto imediato desse trabalho, ou no que é comprado com esse produto de outras nações.” (SMITH, 1776, p. 10)
\end{quote}

Dentro deste prisma, adentramos de fato nesta seção, onde a história até então composta somente por uma imaginação se entrelaçam em um espaço e se encontra com o presente econômico.

\subsection{Dados como a sombra sob a luz da economia.}

Ao mergulharmos na caracterização de uma atividade econômica como pertencente ao setor turístico ocorre quando sua produção principal é composta por bens e serviços cuja demanda é fortemente influenciada pelos visitantes (IBGE, 2006). Diferentemente de outros setores econômicos, cuja mensuração é direta, o turismo é definido a partir da demanda, ou seja, do consumo realizado por não-residentes. É nesse contexto que surge o conceito de \textit{Atividades Características do Turismo} (ACTs), uma ferramenta para isolar e quantificar o impacto econômico do turismo. Embora muitas ACTs, como serviços de alimentação e transporte, também sejam consumidas por residentes, sua inclusão se justifica pela influência significativa da demanda turística, permitindo análises mais precisas sobre a contribuição do setor para o Produto Interno Bruto (PIB) (Ribeiro et al., \cite{ribeiro2021}).

\begin{table}[htbp]
\centering
\caption{Atividades Características do Turismo (IBGE)}
\label{tab:atividades_turismo}
\begin{tabular}{l}
\hline
\textbf{Atividade Característica do Turismo} \\ \hline
Hotéis, pousadas e similares \\
Restaurantes, bares e similares \\
Transporte rodoviário de passageiros \\
Transporte marítimo de passageiros \\
Transporte aéreo de passageiros \\
Serviços auxiliares ao transporte de passageiros \\
Agências de viagens, operadoras e similares \\
Aluguel de veículos e equipamentos de transporte de passageiros \\
Serviços esportivos e de lazer \\ \hline
\end{tabular}
\begin{flushleft}
\footnotesize{Fonte: Instituto Brasileiro de Geografia e Estatística (IBGE), \textit{Economia do turismo: análise das atividades características do turismo – 2003}, p. 14. Adaptado.}
\end{flushleft}
\end{table}


Conceituanddo a tabela \ref{tab:atividades_turismo} o gasto com hospedagem constitui um dos principais componentes do consumo turístico, refletindo a necessidade básica de pernoite do visitante. Entre as atividades características do turismo (ACTs), os serviços de alojamento e alimentação representam pilares centrais da experiência turística e da economia do setor (IBGE). 

A relação dessas atividades com o turismo se evidencia na sazonalidade e na localização geográfica: estabelecimentos situados em áreas de grande fluxo turístico dependem intrinsecamente dessa demanda, e sua expansão ou retração é um indicador direto da saúde do setor.

 
 No presente contexto, o vinho Château Bran Cabernet Sauvignon exemplifica como produtos regionais podem ampliar a experiência do visitante: mais do que um consumo local, o vinho funciona como uma extensão da visita ao patrimônio. Produzido na região vinícola de Dealu Mare e comercializado na loja oficial do Castelo de Bran, ele permite que os turistas vivenciem a cultura local, ao mesmo tempo em que fortalece a economia regional e promove o turismo temático.
 
\begin{figure}[H]
    \centering
    \caption{\centering Vinho Château Bran Cabernet Sauvignon}
    \includegraphics[width=0.4\linewidth]{vinho.png}
    
    \label{fig:vinho_chateau_bran}
    \vspace{2mm}
    {\centering \footnotesize Fonte: CASTELUL BRAN, 2025. Disponível em: \url{https://brancastleshop.com/en/product/vin-cabernet-sauvignon-chateau-bran/}. Acesso em: 28 out. 2025.\par}
\end{figure}


Os serviços de transporte e a intermediação de serviços turísticos constituem a infraestrutura essencial para o acesso e a organização da viagem. O transporte conecta o turista ao destino, sendo um pré-requisito para a atividade turística, enquanto as agências de viagens agregam e facilitam serviços, funcionando como mediadoras da experiência. A saúde financeira dessas ACTs de infraestrutura e intermediação indica a conectividade e a capacidade de distribuição do produto turístico em um país ou região.

Complementando a experiência do visitante, os serviços desportivos e de lazer (incluindo atividades culturais, parques temáticos, entre outros) são frequentemente os motivadores finais da viagem. A relevância dessas atividades está em sua capacidade de transformar a visita em uma experiência memorável, gerando receita e emprego em setores além do básico (alojamento e alimentação), demonstrando a ampla comunhão do turismo na economia.

Como a organização desses dados e de suma importacia a OMT padroniza o dados sobre as ACTS sendo elas trabalhadas amplamente pelo mundo como forma de mensurar o impacto na região, como uma forma de esclarecer com os dados os impactos economicos.



\subsection{Macro e Micro Economia}

A macroeconomia e a microeconomia são como duas lentes que permitem observar a complexa realidade econômica, cada uma em uma escala diferente, mas ambas indispensáveis para compreender o funcionamento das economias modernas.

A \textbf{macroeconomia} (Sandroni, 2019) emerge da análise da economia como um todo, observando os grandes agregados econômicos, como o crescimento do Produto Interno Bruto (PIB), a taxa de desemprego, a inflação e a balança comercial. Ela se dedica a entender as forças que moldam o desempenho econômico global, com foco nas políticas públicas, nas flutuações cíclicas da economia e na interação entre diferentes setores da sociedade. Um exemplo prático dessa perspectiva pode ser encontrado no turismo literário associado à obra Drácula. A popularização da história e a associação do Castelo de Bran com o romance aumentaram significativamente o fluxo de turistas para a região, gerando crescimento econômico local, impacto em múltiplos setores e efeito multiplicador sobre a economia, características típicas da análise macroeconômica.

Em contraste, a \textbf{microeconomia} (Sandroni, 2019) volta-se para os elementos que constituem a base da economia: os indivíduos, as empresas e os mercados. Ela analisa as escolhas e comportamentos dos agentes econômicos em contextos específicos, como o consumo de bens e serviços, a formação de preços e a distribuição de recursos escassos. No romance de Bram Stoker, um exemplo de microeconomia pode ser observado nas decisões de Jonathan Harker ao aceitar viajar para o castelo de Drácula, ou nas trocas que cada personagem realiza com o Conde, enfrentando custos e riscos individuais em busca de algum benefício. Essas escolhas refletem a lógica microeconômica, na qual agentes específicos tomam decisões considerando incentivos e restrições pessoais.

Assim, enquanto a macroeconomia se preocupa com as grandes variáveis que regem a economia nacional e global, a microeconomia investiga os detalhes que sustentam essa grande teia econômica. Ambas, portanto, são peças essenciais para a compreensão da dinâmica econômica, oferecendo diferentes perspectivas sobre as questões que afetam a vida social, o bem-estar coletivo e até mesmo a forma como o turismo literário transforma uma região em cenário econômico relevante.











% ----------------------------------------------------------
% metodologia
% ----------------------------------------------------------

\chapter{Metodologia}

O presente estudo busca compreender de que forma o turismo literário associado à obra \textit{Drácula}, de Bram Stoker, impacta a economia do vilarejo de Bran, na Romênia. A partir dessa problemática, estabelecem-se as seguintes questões centrais: (1) quais são os principais impactos econômicos diretos e indiretos do turismo literário de \textit{Drácula} em Bran; (2) como o mito literário influencia o fluxo turístico e a geração de renda local; (3) de que maneira os setores característicos do turismo  (como hospedagem, alimentação, transporte e lazer)  são afetados; e (4) como o caso de Bran se compara a outros destinos literários internacionais.  

A pesquisa adota uma \textbf{abordagem mista}, combinando métodos \textbf{qualitativos e quantitativos}, para integrar dimensões simbólicas e econômicas do turismo literário. De natureza exploratória e descritiva, esta investigação estrutura-se como um \textbf{estudo de caso}. A base teórico-metodológica segue o que é proposto por Lima Junior, Oliveira, Santos e Schnekenberg (2021), ao apresentarem a \textbf{Análise Documental como percurso metodológico na pesquisa qualitativa}, enfatizando a importância da interpretação de documentos diversos para compreender fenômenos sociais e culturais.  

.  

\section{Pesquisa bibliográfica}

A revisão bibliográfica constitui a base teórica da pesquisa e será conduzida por meio da análise crítica de literatura científica, abrangendo temas que possibilitem compreender o turismo literário sob perspectivas cultural, social e econômica. Foram utilizadas as bases \textit{Google Scholar}, \textit{CAPES}, \textit{SciELO} e a plataforma \textit{Sucupira} para a verificação de qualidade das fontes.  

Os principais eixos teóricos que fundamentam esta pesquisa abrangem quatro dimensões centrais. A primeira refere-se ao turismo literário e aos seus fundamentos culturais, compreendidos como expressão da relação entre narrativa, identidade e território. A segunda aborda os impactos econômicos do turismo, com ênfase nas metodologias de mensuração que permitem avaliar a contribuição dessa atividade para o desenvolvimento local. A terceira dimensão trata da Análise Documental como percurso metodológico, sustentada pelos pressupostos de Sá-Silva, Lüdke e André (1986), que destacam a relevância desse método para a investigação de fenômenos sociais e culturais a partir de registros e fontes documentais. Por fim, a quarta dimensão considera a pesquisa qualitativa e a triangulação metodológica, conforme as orientações de Tuzzo e Braga (2016), que defendem a integração entre diferentes técnicas de análise como meio de ampliar a validade e a profundidade interpretativa dos resultados. Esses referenciais sustentam a compreensão do turismo literário como prática social e cultural e orientam a análise dos documentos oficiais e dados secundários utilizados no estudo.  

\section{Análise documental}

A segunda etapa da metodologia consiste na \textbf{Análise Documental}, fundamentada nos conceitos de Cellard (2008) e Sá-Silva et al. (2009), que definem esse método como o exame sistemático de documentos de diferentes naturezas, a fim de extrair informações significativas relacionadas ao objeto de estudo. Conforme Lima Junior et al. (2021), a Análise Documental utiliza procedimentos técnicos e científicos específicos para compreender o conteúdo dos documentos e deles obter informações relevantes conforme os objetivos da pesquisa.  

Serão analisados documentos oficiais, relatórios governamentais e bases estatísticas que tratam da atividade turística na Romênia, com ênfase na região de Bran. A coleta de dados quantitativos será realizada a partir de fontes oficiais, entre as quais se destacam o Instituto Nacional de Estatística da Romênia (INSSE), responsável pelas séries históricas sobre fluxo turístico e indicadores econômicos; a Prefeitura de Bran (Primăria Comunei Bran), que disponibiliza informações locais sobre receitas e gestão turística; a Organização Mundial do Turismo (OMT), que fornece parâmetros comparativos internacionais; e o Ministério da Economia e Turismo da Romênia, cujos relatórios oferecem dados sobre políticas públicas e desempenho do setor. Essas fontes permitirão examinar, de forma integrada, aspectos econômicos, institucionais e sociais que caracterizam o turismo literário no contexto estudado.


A análise qualitativa seguirá as orientações de Cechinel et al. (2016), que destacam a importância da crítica dos documentos quanto ao contexto, autoria, confiabilidade e relevância para o tema estudado. Dessa forma, cada documento será examinado de maneira criteriosa, considerando o contexto em que foi produzido, os autores responsáveis e sua credibilidade, a natureza e a finalidade do material, bem como os conceitos-chave e as informações que possam contribuir para sustentar a discussão teórica e empírica da pesquisa.

\section{Integração das abordagens qualitativa e quantitativa}

Com base em Minayo (2009) e Creswell (2007), a abordagem qualitativa busca compreender os significados e valores atribuídos ao fenômeno do turismo literário, enquanto a quantitativa contribui com dados objetivos e mensuráveis, permitindo identificar padrões e tendências econômicas. Essa integração metodológica conhecida como triangulação, é defendida por Tuzzo e Braga (2016) como forma de ampliar a validade e a profundidade interpretativa da pesquisa.  

A partir dessa combinação, o estudo analisará indicadores de fluxo turístico, ocupação hoteleira, sazonalidade e receita gerada, relacionando-os às representações simbólicas e culturais associadas à figura de \textit{Drácula}.  

\section{Fundamentação do método}

A adoção da Análise Documental como principal técnica qualitativa justifica-se pela sua capacidade de reunir e interpretar informações históricas, culturais e econômicas a partir de documentos diversificados, oferecendo uma visão contextualizada do objeto de estudo. Autores como Lüdke e André (1986), destacam que a análise documental é especialmente relevante quando o pesquisador busca compreender fenômenos sociais a partir de registros já existentes, o que se aplica ao presente estudo, dada a disponibilidade de dados oficiais e registros históricos sobre o turismo em Bran.  

Dessa forma, a metodologia proposta combina a robustez dos dados quantitativos oficiais com a profundidade interpretativa da abordagem qualitativa, permitindo avaliar o turismo literário de \textit{Drácula} como fenômeno cultural e econômico. Assim, busca-se compreender não apenas os números que sustentam a atividade turística, mas também os significados simbólicos e identitários que transformaram Bran em um dos destinos literários mais vampiresco da Europa.











\chapter{Estudo de Caso}
\section{Bran no Espelho: A Construção do Turismo}  
A identificação das Atividades Características do Turismo (ACTs) em Bran é essencial para entender como o turismo literário se integra à economia local, que, como já discutido, depende fortemente do turismo proveniente de Brașov. Nesse contexto, os serviços de hospedagem, alimentação, transporte e lazer são os principais pilares da economia da comuna. O transporte aqui desempenha um papel estratégico, pois, devido ao tamanho pequeno e à infraestrutura urbana limitada de Bran, a localidade depende das conexões proporcionadas pelo condado de Brașov para garantir o fluxo constante de turistas. Sendo assim as rotas rodoviárias que ligam Bran a Brașov e Bucareste integram a comuna ao circuito turístico regional, facilitando o deslocamento dos visitantes e refletindo a vitalidade econômica do setor.

Segundo dados do Instituto Nacional de Estatística da Romênia (INSSE, 2025), o condado de Brașov consolidou-se, em 2023, como um dos principais destinos turísticos do país, ocupando a terceira posição nacional em número de turistas hospedados, atrás apenas de Bucareste e Constanța. O total de visitantes que pernoitaram na região cresceu 10,8\% em relação a 2022, enquanto o número de pernoites aumentou 12,1\%. Desde 2018, o número de estabelecimentos de hospedagem expandiu-se 30,4\%, posicionando Brașov em segundo lugar nacional em capacidade de acolhimento.

Esta seção analisa a sazonalidade do turismo literário na região de Bran e seus impactos econômicos, considerando indicadores como fluxo de visitantes e ocupação hoteleira. A partir desses parâmetros, busca-se compreender as variações temporais da demanda turística e seus reflexos sobre a economia regional, destacando o papel das ACTs na estrutura produtiva local. Uma vez que Bran recebe a maior parte de seus turistas provenientes de Brașov, torna-se fácil mensurar o fluxo de visitantes que seguem essa rota.

Começaremos pela Tabela \ref{tab:estruturas_turisticas} que demonstra a ampliação das estruturas de hospedagem, entre 2018 e 2023, com destaque para o aumento expressivo de apartamentos e quartos para aluguel, que refletem o crescimento do turismo alternativo e da economia de compartilhamento na região. O incremento de unidades hoteleiras e pousadas rurais evidencia a diversificação da oferta de serviços e a adaptação do setor às novas demandas dos visitantes, reforçando o papel das ACTs como base do desenvolvimento turístico local.


\begin{table}[H]
\centering
\caption{Estruturas de alojamento turístico com função de hospedagem(2018–2023)}
\label{tab:estruturas_turisticas}
\begin{tabular}{lcccccc}
\hline
\textbf{Tipo de unidade de alojamento} & \textbf{2018} & \textbf{2019} & \textbf{2020} & \textbf{2021} & \textbf{2022} & \textbf{2023} \\ 
\hline
Unidades de alojamento – total & 961 & 930 & 902 & 1207 & 1229 & 1253 \\
Hotéis & 127 & 125 & 118 & 121 & 120 & 121 \\
Hostels & 26 & 26 & 25 & 25 & 23 & 19 \\
Apartamentos e quartos para alugar & – & – & – & 306 & 350 & 398 \\
Hotéis-apartamento & 2 & 1 & 3 & 3 & 3 & 2 \\
Motéis & 9 & 9 & 9 & 9 & 9 & 9 \\
Vilas turísticas & 84 & 82 & 77 & 69 & 64 & 57 \\
Chalés turísticos & 34 & 36 & 36 & 33 & 34 & 34 \\
Bangalôs & 5 & 5 & 5 & 6 & 6 & 11 \\
Vilas de férias & 2 & 2 & 2 & 1 & 1 & 2 \\
Campings & 2 & 2 & 2 & 2 & 6 & 6 \\
Casas turísticas & 3 & 3 & 3 & 4 & 2 & 4 \\
Acampamentos escolares & 1 & 1 & 1 & 1 & 1 & 1 \\
Pousadas turísticas & 264 & 255 & 242 & 240 & 230 & 217 \\
Pousadas rurais & 402 & 383 & 379 & 387 & 380 & 372 \\
\hline
\end{tabular}
\begin{flushleft}
\footnotesize \textbf{Fonte:} Instituto Nacional de Estatística da Romênia (2024). Dados adaptados do Capítulo 15 – Turismo, Condado de Brașov.
\end{flushleft}
\end{table}

E de suma importancia observar aqui que nem todos os tipos de alojamentos sofreram aumento, para calcular o aumento percentual no número de unidades de alojamento turístico entre 2018 e 2023, utilizamos a seguinte fórmula de crescimento percentual:

\[
\text{Aumento percentual} = \frac{\text{Valor final} - \text{Valor inicial}}{\text{Valor inicial}} \times 100
\]

Essa fórmula nos ajuda a medir a variação percentual entre os valores de 2018 e 2023, permitindo avaliar mais criticamente o crescimento ou diminuição de cada tipo de unidade de hospedagem ao longo dos anos.

\newpage
\textbf{Cálculos:}

1. Unidades de Alojamento - Total:

   Valor inicial (2018): 961 - 
   Valor final (2023): 1253

   \[
   \text{Aumento percentual} = \frac{1253 - 961}{961} \times 100 = 30,4\%
   \]

O número total de unidades de alojamento aumentou 30,4\% de 2018 a 2023, indicando um crescimento significativo na oferta de hospedagem em Brașov.

2. Hotéis:
Valor inicial (2018): 127  - Valor final (2023): 121

   \[
   \text{Aumento percentual} = \frac{121 - 127}{127} \times 100 = -4,7\%
   \]

O número de hotéis teve uma diminuição de 4,7\% ao longo dos anos, sugerindo uma possível estagnação ou menor atratividade para esse tipo de hospedagem em comparação com outros tipos.

3. Hostels:

   Valor inicial (2018): 26 -  Valor final (2023): 19

   \[
   \text{Aumento percentual} = \frac{19 - 26}{26} \times 100 = -26,9\%
   \]

O número de hostels diminuiu em 26,9\%, refletindo uma queda significativa na oferta deste tipo de hospedagem, possivelmente devido a mudanças nas preferências dos turistas ou uma saturação do mercado.

4. Apartamentos e Quartos para Alugar:

   Valor inicial (2018): 0 -  Valor final (2023): 398

Não é possível calcular um aumento percentual devido ao valor inicial ser zero, mas observamos um aumento expressivo no número de apartamentos e quartos para alugar. Este crescimento de 398 unidades em 2023 reflete a popularização do turismo alternativo e da economia de compartilhamento (como o Airbnb).

5. Pousadas Rurais:

   Valor inicial (2018): 402 - Valor final (2023): 372

   \[
   \text{Aumento percentual} = \frac{372 - 402}{402} \times 100 = -7,5\%
   \]


Para criar uma conclusão concisa, nossa próxima análise busca entender a evolução da capacidade de alojamento turístico em Brașov entre 2018 e 2023. A Tabela \ref{tab:capacidade_existente} revela que, embora o número total de unidades de alojamento tenha mostrado um crescimento modesto de 29.832 em 2018 para 34.421 em 2023, o aumento não foi uniforme entre as categorias.


\begin{table}[H]
\centering
\caption{Capacidade de alojamento turístico existente – 31 de julho (2018–2023)}
\label{tab:capacidade_existente}
\begin{tabular}{lcccccc}
\hline
\textbf{Tipo de unidade de alojamento} & \textbf{2018} & \textbf{2019} & \textbf{2020} & \textbf{2021} & \textbf{2022} & \textbf{2023} \\ 
\hline
Unidades de alojamento – total & 29\,832 & 29\,438 & 28\,726 & 33\,550 & 34\,169 & 34\,421 \\
Hotéis & 11\,928 & 11\,841 & 11\,369 & 11\,505 & 11\,648 & 11\,864 \\
Hostels & 965 & 885 & 954 & 1\,074 & 1\,003 & 732 \\
Apartamentos e quartos para alugar & – & – & – & 4\,663 & 5\,515 & 6\,201 \\
Hotéis-apartamento & 52 & 12 & 116 & 112 & 108 & 60 \\
Motéis & 630 & 594 & 630 & 630 & 630 & 630 \\
Vilas turísticas & 1\,853 & 1\,804 & 1\,787 & 1\,661 & 1\,497 & 1\,291 \\
Chalés turísticos & 1\,091 & 1\,161 & 1\,165 & 1\,113 & 1\,188 & 1\,152 \\
Bangalôs & 118 & 118 & 118 & 130 & 130 & 130 \\
Vilas de férias & 120 & 120 & 120 & 48 & 48 & 266 \\
Campings & 452 & 452 & 452 & 452 & 728 & 614 \\
Casas turísticas & 96 & 96 & 96 & 146 & 84 & 146 \\
Acampamentos escolares & 50 & 74 & 74 & 74 & 74 & 74 \\
Pousadas turísticas & 5\,583 & 5\,518 & 5\,150 & 5\,022 & 4\,828 & 4\,545 \\
Pousadas rurais & 6\,894 & 6\,763 & 6\,695 & 6\,920 & 6\,688 & 6\,716 \\
\hline
\end{tabular}
\begin{flushleft}
\footnotesize \textbf{Fonte:} Instituto Nacional de Estatística da Romênia (2024). Dados adaptados do Capítulo 15 – Turismo, Condado de Brașov.
\end{flushleft}
\end{table}

Os Hotéis aqui mantiveram-se praticamente estáveis, com uma leve expansão de 11.928 para 11.864 unidades, enquanto as Pousadas turísticas e Vilas turísticas apresentaram quedas significativas, de 5.583 para 4.545 e de 1.853 para 1.291 unidades, respectivamente. Por outro lado, o crescimento explosivo de Apartamentos e quartos para alugar (de 0 para 6.201 unidades) reflete uma mudança na oferta, destacando-se como uma categoria em expansão.

Esses dados indicam uma transformação na estrutura do mercado, onde, apesar da estabilidade no número de grandes estabelecimentos, houve uma diversificação significativa com o aumento de unidades alternativas, como os apartamentos. A análise desses números revela que a estratégia de crescimento do setor turístico de Brașov se concentrada na diversificação e no fortalecimento de modalidades mais flexíveis e acessíveis.

Antes de prosseguirmos com a próxima análise, apresento a citação direta de Butler (1980, p. 102), com o objetivo de proporcionar ao leitor uma reflexão sobre os dados apresentados:

\begin{quote}
\textit{O ciclo de vida de um produto turístico compreende as seguintes fases: exploração, investimento, desenvolvimento, consolidação, estagnação e declínio ou revitalização. A sua utilização como instrumento do planejamento turístico se justifica na determinação da fase em que se encontra a localidade em estudo, e nas medidas cabíveis para direcionar seu desenvolvimento.}
\end{quote}

Estudada a sazonalidade entre os anos por meio dos dados de hospedagem, prossegue-se agora para a análise de outra \textit{Atividade Característica do Turismo}: o fluxo de turistas.

\begin{table}
\centering
\caption{Chegadas de visitantes estrangeiros no condado de Brașov, por país de origem (2018–2023)}
\label{tab:visitantes_estrangeiros}
\begin{tabular}{lcccccc}
\hline
\textbf{País / Região} & \textbf{2018} & \textbf{2019} & \textbf{2020} & \textbf{2021} & \textbf{2022} & \textbf{2023} \\
\hline
\textbf{Total geral} & 207\,488 & 191\,165 & 26\,685 & 54\,399 & 107\,595 & 166\,627 \\
\hline
\textbf{Europa – total} & 149\,487 & 137\,264 & 21\,990 & 44\,203 & 77\,716 & 124\,318 \\
União Europeia (total) & 126\,632 & 113\,254 & 14\,599 & 32\,770 & 52\,940 & 85\,397 \\
Alemanha & 25\,585 & 22\,926 & 3\,745 & 7\,563 & 13\,220 & 18\,845 \\
Itália & 11\,654 & 10\,286 & 1\,431 & 3\,287 & 5\,258 & 8\,600 \\
Espanha & 15\,821 & 15\,519 & 991 & 3\,143 & 4\,689 & 8\,543 \\
Polônia & 13\,304 & 8\,469 & 1\,193 & 4\,378 & 5\,116 & 8\,860 \\
França & 10\,277 & 9\,454 & 1\,753 & 3\,704 & 5\,168 & 7\,469 \\
Reino Unido & 13\,441 & 11\,449 & – & – & – & – \\
Bulgária & 5\,089 & 4\,480 & 709 & 1\,491 & 3\,602 & 7\,409 \\
Grécia & 3\,082 & 2\,696 & 598 & 631 & 1\,572 & 1\,956 \\
Bélgica & 2\,599 & 2\,774 & 393 & 1\,230 & 1\,555 & 3\,287 \\
Países Baixos & 4\,232 & 2\,716 & 497 & 1\,255 & 1\,948 & 4\,219 \\
Portugal & 1\,634 & 1\,400 & 84 & 264 & 483 & 748 \\
Outros países europeus & 10\,479 & 9\,163 & 1\,308 & 3\,941 & 6\,065 & 8\,972 \\
\hline
\textbf{Países não pertencentes à UE} & 22\,855 & 24\,010 & 7\,391 & 11\,433 & 24\,776 & 38\,921 \\
República da Moldávia & 7\,543 & 7\,168 & 2\,396 & 2\,774 & 7\,268 & 12\,676 \\
Reino Unido (não-UE pós-Brexit) & – & – & 2\,118 & 2\,833 & 7\,154 & 9\,858 \\
Suíça & 1\,862 & 1\,656 & 203 & 624 & 974 & 2\,614 \\
Turquia & 1\,716 & 1\,804 & 383 & 639 & 1\,342 & 2\,005 \\
Ucrânia & 3\,814 & 5\,266 & 923 & 1\,799 & 5\,450 & 7\,902 \\
Sérvia & 1\,074 & 896 & 202 & 200 & 715 & 1\,228 \\
\hline
\textbf{África} & 594 & 597 & 222 & 224 & 471 & 569 \\
Egito & 107 & 82 & 20 & 28 & 54 & 90 \\
Marrocos & 29 & 21 & 9 & 26 & 12 & 56 \\
Tunísia & 30 & 82 & 103 & 16 & 80 & 85 \\
\hline
\textbf{América do Norte} & 14\,417 & 13\,583 & 1\,135 & 3\,517 & 8\,390 & 13\,870 \\
EUA & 12\,475 & 11\,609 & 964 & 3\,168 & 7\,459 & 12\,215 \\
Canadá & 1\,942 & 1\,974 & 171 & 349 & 931 & 1\,655 \\
\hline
\textbf{América do Sul e Central} & 1\,736 & 1\,870 & 177 & 289 & 679 & 1\,337 \\
Brasil & 727 & 642 & 55 & 54 & 176 & 377 \\
México & 333 & 415 & 26 & 72 & 252 & 390 \\
Argentina & 203 & 357 & 17 & 29 & 68 & 108 \\
\hline
\textbf{Ásia} & 36\,204 & 35\,088 & 2\,934 & 5\,612 & 16\,425 & 23\,619 \\
Israel & 24\,895 & 23\,316 & 1\,993 & 4\,499 & 14\,095 & 17\,358 \\
China & 4\,700 & 5\,700 & 240 & 161 & 375 & 1\,788 \\
Japão & 2\,658 & 1\,988 & 235 & 52 & 173 & 466 \\
Índia & 595 & 701 & 69 & 79 & 291 & 466 \\
Irã & 121 & 304 & 28 & 42 & 122 & 549 \\
\hline
\textbf{Oceania} & 2\,002 & 2\,257 & 135 & 169 & 553 & 1\,721 \\
Austrália & 1\,567 & 1\,835 & 92 & 88 & 427 & 1\,429 \\
Nova Zelândia & 305 & 331 & 23 & 40 & 61 & 137 \\
\hline
\textbf{Não especificados} & 3\,048 & 506 & 92 & 385 & 3\,361 & 1\,193 \\
\hline
\end{tabular}
\begin{flushleft}
\footnotesize \textbf{Fonte:} Instituto Nacional de Estatística da Romênia (2024). Dados adaptados do Capítulo 15 – \textit{Turismo}, Condado de Brașov.
\end{flushleft}
\end{table}



A Tabela \ref{tab:visitantes_estrangeiros} evidencia o crescimento expressivo do turismo internacional no condado de Brașov, com destaque para o retorno gradual observado após a pandemia de COVID-19. O aumento no número de visitantes estrangeiros, especialmente oriundos da Europa Ocidental e da América do Norte, demonstra o fortalecimento da imagem da Transilvânia como um destino literário e cultural. A presença marcante de turistas provenientes de países como Alemanha, Itália e Espanha reforça a relevância do turismo literário associado ao mito de \textit{Drácula} no contexto europeu contemporâneo.

Mas precisamos retomar brevemente aqui para o impacto do COVID-19, por meio da variação absoluta e percentual nas chegadas de visitantes estrangeiros entre os anos de 2019 e 2020 analizaremos esse impacto pelo crescimento absoluto e percentual:
    \[
    \text{Crescimento Absoluto} = \text{Valor no ano atual} - \text{Valor no ano anterior}
    \]
    \[
    \text{Crescimento Percentual} = \left( \frac{\text{Crescimento Absoluto}}{\text{Valor no ano anterior}} \right) \times 100
    \]


A Tabela \ref{tab:visitantes_estrangeiros} apresenta os totais de visitantes estrangeiros no condado de Brașov para os anos de 2019 e 2020, sendo respectivamente 191.165 visitantes em 2019 e 26.685 visitantes em 2020.


Agora, aplicando as fórmulas:
   
    \[
    \text{Crescimento Absoluto} = 26\,685 - 191\,165 = -164\,480
    \]
    
    
    \[
    \text{Crescimento Percentual} = \left( \frac{-164\,480}{191\,165} \right) \times 100 = -85,9\%
    \]

A análise do crescimento percentual revela um impacto profundo da pandemia de COVID-19 no turismo local. Em 2020, o número de visitantes caiu 85,9\% em relação a 2019, uma redução drástica que pode ser atribuída a restrições de viagem, fechamento de fronteiras, e o receio das pessoas em viajar devido ao risco de infecção. 

Para viabilizar nossa análise de forma justa e representativa, excluímos os dados de 2019 e 2020, anos fortemente impactados pela pandemia de COVID-19, que causou uma queda abrupta nas chegadas de turistas. A inclusão desses anos no cálculo poderia distorcer os resultados, pois a pandemia alterou drasticamente os padrões de viagem, o que não reflete os comportamentos típicos do setor turístico. Portanto, ao focarmos nos dados de 2018, 2021, 2022 e 2023, garantimos uma análise mais equilibrada, que reflete as tendências normais do fluxo turístico, sem o viés das condições excepcionais impostas pela crise sanitária global.

Esse recorte temporal assegura que a comparação entre as regiões seja mais fiel ao contexto pré-pandemia e à recuperação observada nos anos subsequentes, proporcionando uma avaliação mais precisa da distribuição geográfica do turismo. Dessa forma, evitamos que os resultados da análise sejam influenciados por anomalias causadas por fatores externos, permitindo uma leitura mais justa do impacto de fatores econômicos, culturais e regionais no fluxo de turistas. Sendo assim proseguiremos a analise o total de visitantes de 2018, 2021, 2022 e 2023 é calculado da seguinte forma:
\[
\text{Total de Visitantes (Excluindo 2019 e 2020)} = 207\,488 + 54\,399 + 107\,595 + 166\,627 = 536\,109
\]

\textbf{Resultados das percentual\footnote{A fórmula para o cálculo percentual de visitantes por região foi apresentada anteriormente.} por Região}
\begin{itemize}
    \item \textbf{Europa:}
        \begin{itemize}
            \item Visitantes da Europa em 2018, 2021, 2022, 2023: 395\,724
            \item Proporção:  
            \[
            \frac{395\,724}{536\,109} \times 100 = 73,8\%
            \]
        \end{itemize}
    
    \item \textbf{América do Norte:}
        \begin{itemize}
            \item Visitantes da América do Norte em 2018, 2021, 2022, 2023: 40\,194
            \item Proporção:  
            \[
            \frac{40\,194}{536\,109} \times 100 = 7,5\%
            \]
        \end{itemize}
   
    \item \textbf{Ásia:}
        \begin{itemize}
            \item Visitantes da Ásia em 2018, 2021, 2022, 2023: 81\,860
            \item Proporção:  
            \[
            \frac{81\,860}{536\,109} \times 100 = 15,3\%
            \]
        \end{itemize}
        
\end{itemize}


A análise revela que, excluindo 2019 e 2020, a Europa ainda representa a maior parte do turismo no condado de Brașov, com 73,8\% dos visitantes. A Ásia e a América do Norte mantêm uma participação relevante, com 15,3\% e 7,5\%, respectivamente, destacando-se como regiões com uma contribuição considerável para o turismo da região, embora com uma proporção inferior à da Europa.


A Figura 2 apresentada mostra o tempo médio de permanência no Condado de Brașov ao longo do período de janeiro de 2022 até julho de 2025, entre romenos (Români), estrangeiros (Străini) e o total de turistas. No gráfico, observa-se que os turistas estrangeiros costumam permanecer mais tempo no condado, com valores próximos ou acima de 2,5 dias em diversos períodos, enquanto os turistas romenos apresentam um tempo médio de permanência mais estável, geralmente em torno de 2 dias. A linha referente ao total de turistas se situa entre essas duas curvas, refletindo a média geral.

\begin{figure}[h!]
    \centering
    \caption{Duração média do tempo de estadia em Brașov, dados de 2025.}
    \includegraphics[width=1\linewidth]{tempo-estadia.png}
    
    \label{fig:durata_sejur}
    \vspace{2mm}
    {\footnotesize Fonte: Instituto Nacional de Estatística da Romênia – Brașov (2025). Disponível em: \url{https://brasov.insse.ro/comunicate-de-presa/turism/turism/#respond}. Acesso em: 28 out. 2025.}
\end{figure}

Essa análise se mostra relevante porque, ao cruzarmos os dados da imagem sobre o tempo médio de permanência com as informações da tabela sobre a origem dos visitantes, podemos observar uma correlação entre a origem geográfica dos turistas e a duração de sua estadia em Brașov. A imagem destaca que os turistas estrangeiros tendem a permanecer por períodos mais longos, frequentemente acima de 2,5 dias, enquanto os turistas romenos apresentam um tempo de permanência mais estável, em torno de 2 dias.

\newpage
\section{Brașov e Bran: O Lugar Onde o Sangue Vira Ouro}
O investimento, em seu sentido mais amplo, é um motor essencial para o desenvolvimento e crescimento econômico de uma região. Embora o investimento direto nas \textit{Atividades Características do Turismo} seja crucial, o investimento em outros setores da economia desempenha um papel igualmente fundamental, gerando um impacto indireto que molda a experiência do turista e a competitividade do destino.

Dwyer, Forsyth e Spurr (2004) afirmam que os impactos econômicos do turismo não se limitam ao setor diretamente relacionado, como hospedagem e transporte, mas também influenciam setores mais amplos da economia, como comércio, infraestrutura e serviços públicos. A robustez econômica geral, impulsionada por investimentos em áreas como a indústria de transformação, comércio ou administração pública, cria um ambiente de negócios mais estável e próspero. Esse dinamismo se traduz em melhorias na infraestrutura básica, maior disponibilidade de serviços de apoio e um aumento na renda e qualidade de vida da população local, fatores que, em última instância, beneficiam o setor de turismo.


\subsection{O Impacto dos Investimentos em Brașov e como isso afeta os turistas?}


A análise da distribuição dos investimentos líquidos por domínio de atividade, como ilustrado na Tabela \ref{tab:investimentos_brasov}, permite identificar a estrutura econômica de uma região e inferir seu potencial de impacto indireto no turismo. Dwyer, Forsyth e Spurr (2004) destacam que, embora o investimento direto em hotéis e restaurantes (uma ACT) seja vital para a oferta imediata, a competitividade de longo prazo do destino dependem da capacidade da economia local de absorver e gerar valor em múltiplos setores. 

\begin{table}[H]
\centering
\caption{Estrutura dos Investimentos Líquidos por Domínio de Atividade — Condado de Brașov (Semestre I de 2025)}
\label{tab:investimentos_brasov}
\begin{tabular}{lcc}
\hline
\textbf{Domínio de Atividade} & \textbf{2025} & \textbf{2024} \\ \hline
Indústria de Transformação & 56,8\% & 53,9\% \\
Administração Pública e Defesa & 17,3\% & 17,1\% \\
Comércio (Atacado e Varejo) & 7,4\% & 11,1\% \\
Saúde e Assistência Social & 4,0\% & 2,8\% \\
Distribuição de Água e Saneamento & 3,5\% & 2,2\% \\
Construção & 3,9\% & 4,6\% \\
Transporte e Armazenamento & 2,4\% & 2,3\% \\
Hotéis e Restaurantes & 1,2\% & 0,7\% \\
Transações Imobiliárias & 1,9\% & 0,2\% \\
Agricultura, Silvicultura e Pesca & 0,4\% & 1,2\% \\
Outros & 1,2\% & 1,9\% \\ \hline
\textbf{Total} & \textbf{100,0\%} & \textbf{100,0\%} \\ \hline
\end{tabular}

\vspace{2mm}
{\footnotesize Fonte: Direcţia Judeţeană de Statistică Braşov (DJS Braşov), “Investiţii ‒ Judeţul Braşov, 2025”. Tradução nossa. - Adaptado}
\end{table}


A tabela \ref{tab:investimentos_brasov} mostra a distribuição dos investimentos líquidos por domínio de atividade em Brașov no primeiro semestre de 2025, destacando uma maior concentração de recursos em indústria de transformação (56,8\%) e administração pública e defesa (17,3\%). Esse tipo de investimento, embora não diretamente relacionado às atividades características do turismo, fortalece a base econômica da região, alinhando-se com o argumento de Dwyer, Forsyth e Spurr (2004), de que os investimentos em setores não turísticos, como infraestrutura e indústria, são fundamentais para criar um ambiente propício ao turismo, como discutido anteriormente.

\subsection{O PIB Romeno}

O turismo continua a desempenhar um papel estratégico na economia da Romênia, não apenas como um setor de serviços, mas também como um motor de crescimento econômico. De acordo com o World Travel and Tourism Council (WTTC, 2023), o setor de turismo na Romênia teve um impacto significativo no Produto Interno Bruto (PIB), com uma contribuição estimada de 3,6\% para o PIB em 2022. Isso representa um aumento considerável em relação aos anos anteriores, refletindo a recuperação do setor após a pandemia.

No cenário macroeconômico, o turismo se torna uma força crescente, suas raízes se aprofundam cada vez mais na estrutura do país. Em 2016, sua contribuição para o PIB foi de 2,765\%, conforme apontado por Comănescu (2019). Porém, dados mais recentes, do ano de 2022, revelam que essa participação ascendeu para 3,6\% (WTTC, 2023), demonstrando não apenas uma recuperação, mas uma expansão sólida. Esse avanço se reflete também em termos monetários: o turismo gerou, em 2022, cerca de 22.000 milhões de RON\footnote{RON (Romanian Leu) é a moeda oficial da Romênia. O "Leu" (plural: "Lei") é subdividido em 100bani (a unidade menor). A moeda foi introduzida em 1867, mas a versão atual do Leu foi adotada em 2005, substituindo a moeda anterior.}, cerca de 4,5 bilhões de euros, consolidando-se como uma das fontes mais consistentes de crescimento econômico (Eurostat, 2023). A balança de pagamentos do país também se beneficiou do turismo internacional, que, em 2022, contribuiu com 6.900 milhões de RON, reforçando o papel do turismo como gerador vital de divisas e equilíbrio econômico (Comănescu, 2023).

O turismo também desempenha um papel vital no sistema fiscal do país, gerando impostos e taxas que são direcionados ao orçamento estatal. Além disso, o setor impulsiona investimentos públicos e privados em infraestrutura, como estradas, aeroportos e redes de comunicação, o que possui um efeito multiplicador positivo na economia nacional (WTTC, 2023).

No nível microeconômico, o turismo é um motor de emprego e renda. Em 2022, o setor gerou diretamente 250.000 postos de trabalho e sustenta empregos indiretos em áreas como serviços de alimentação, transporte e vendas (WTTC, 2023). A maior parte dos gastos dos turistas foi direcionada para acomodação e alimentação, com 44,6\% e 25,2\%, respectivamente (Eurostat, 2023). Isso impulsiona as economias locais, especialmente em regiões turísticas como Bran, um destino popular devido ao seu patrimônio cultural e atrações turísticas.

No entanto, a economia informal ainda representa um desafio no setor, com uma parte significativa das acomodações não registradas oficialmente. Estima-se que isso distorça os dados e afete a arrecadação fiscal, um problema que limita o potencial de crescimento do setor (Comănescu, 2023). A formalização da economia informal é um dos principais desafios para aumentar a transparência e otimizar o crescimento do setor.



\section{Bran no Espelho de Outros Destinos Literários}
A Casa de Anne Frank, localizada em Amsterdã, é o parâmetro comparativo que iremos utilizar nesse estudo para podermos entender mais dos impactos econômicos que circundam a atividade turística nos países da Romênia e Holanda. Aremis Santos (2020) relata em seu trabalho que, após tanto tempo do enaltecimento das narrativas de Anne Frank, em sua literatura autoral de seu diário, a memória do lugar tornou o anexo secreto em que Anne vivia em um ponto de visitação turística, assim como o campo de concentração Bergen-Belsen para o qual as irmãs Frank foram levadas e executadas.

Como mencionado anteriormente, a literatura de Anne Frank pode ser relacionada como uma amostra de dark tourism partindo do pressuposto de que a literatura permite um diálogo entre outras disciplinas, incluindo o turismo. Como explicado também por Santos (2020), é classificado como Thanaturismo no caso do Diário de Anne Frank, que seria:

\begin{quote}
“Prática turística onde há visitas, intencional ou por outras razões, a locais que possam oferecer a representação da morte ou do sofrimento” – Adaptado, Santos (2020)
\end{quote}

A Holanda é um dos países presentes nos relatórios da OCDE (Organização para a Cooperação e Desenvolvimento Econômico) e, de acordo com um de seus documentos, OECD (2016), a atividade turística doméstica e de entrada de visitantes vem crescendo de maneira significativa, excedendo a economia local. Isso demonstra que a taxa econômica relacionada ao turismo e os números da taxa de crescimento empregatício foram vastos. Em 2014, a OCDE demonstra que a Holanda teve recorde de visitantes internacionais, comparada a outros países do grupo, o que é visível pela grande participação de manutenção, investimento e promoção turística pelo governo. 

Apesar disso, não foi possível encontrar informações claras sobre a influência econômica do turismo e dos principais atrativos turísticos da Holanda em seu site. Utilizamos então as informações disponibilizadas no site oficial da Casa-museu Anne Frank (\href{https://www.annefrank.org/en/about-us/annual-report-2024/}{Relatório Anual de 2024}) que conta, neste link, com um relatório anual de visitantes, assim como os infográficos oficiais e dinamizados, que utilizam de base de dados original e atualizada de acordo com as últimas informações oficiais disponibilizadas ao órgão (\href{https://www.untourism.int/tourism-statistics/tourism-data-macroeconomic-indicators}{UN Tourism Statistics} e \href{https://www.untourism.int/tourism-statistics/tourism-data-macroeconomic-indicators}{UN Tourism Indicators}). A organização oficial da Casa-museu de Anne Frank demonstra que, no ano de 2024, o total de visitantes foi de 1.208.327 pessoas, sendo 28\% desses visitantes vindos dos Estados Unidos.

Na Figura \ref{fig:imagem-1} a seguir, é apontado um comparativo de chegada de turistas internacionais por sazonalidade, dos anos de 2021 a 2023, já comparados entre a Romênia e a Holanda. Podemos perceber que as variações possuem um ondulamento semelhante, mas dentro do segundo gráfico, as variações entre os anos são maiores, dando destaque para o maior número de visitantes de julho a outubro na Romênia e de março a dezembro na Holanda.

\begin{figure}[H]
    \centering
    \includegraphics[width=1\linewidth]{imagem-1.png}
    \caption{Fonte: Dados oficiais de turismo. Disponível em: \url{https://www.untourism.int/tourism-statistics/tourism-statistics database.} Acesso em 22 de nov. De 2025. }
    \label{fig:imagem-1}
\end{figure}

Na figura 6 e 7 a seguir são apontados os comparativos entre os dois países em 
questão sobre a receita gerada por turistas internacionais e a chegada de turistas 
internacionais. Nota-se que as ondulações dos gráficos são semelhantes ao passar 
dos anos, dando destaque para a influência da pandemia na variação desses 
indicadores, tanto econômico quanto de fluxo. Mas podemos perceber que a influência 
econômica do mercado turístico é maior para a Romênia do que para a Holanda. 



\begin{figure}[H]
    \centering
    \includegraphics[width=1\linewidth]{image-2.png}
    \caption{Fonte: Dados oficiais de turismo. Disponível em: \url{https://www.untourism.int/tourism-statistics/tourism-statistics-database}. Acesso em: 23 nov. 2025.}
    \label{fig:imagem-2}
\end{figure}

\begin{figure}[H]
    \centering
    \includegraphics[width=1\linewidth]{image-3.png}
    \label{fig:imagem-3}
    \caption{Fonte: Dados oficiais de turismo. Disponível em: \url{https://www.untourism.int/tourism-statistics/tourism-statistics-database}. Acesso em: 23 nov. 2025.}
\end{figure}


Ademais, o próximo gráfico demonstra o indicador econômico de “Tourism GDP by 
Region” e ao lado na mesma imagem, “Tourism GDP by Region and Country”. Esse 
indicador, em tradução livre, diz respeito ao “Gross Domestic Product”, ou seja, o PIB 
per-capta em relação ao turismo, por região no globo terrestre (que foi escolhido para 
esse trabalho somente a Europa) e por países, que engloba os países de estudo aqui 
presentes. Nesse segundo gráfico da imagem, podemos notar que para a Holanda, de 
acordo com as últimas informações disponíveis, o PIB turístico foi de 2,5\% em 2021 e 
na Romênia de 3,0\%.

\begin{figure}[H]
    \centering
    \includegraphics[width=1\linewidth]{image-4.png}
    \caption{Fonte: Dados oficiais de turismo. Disponível em: \url{https://www.untourism.int/tourism-statistics/tourism-statistics-database}. Acesso em: 22 nov. 2025.}
    \label{fig:img-4}
\end{figure}

Nas próximas duas figuras temos então, a indicação do PIB (GDP) direto do turismo, 
levando em consideração o PIB total de cada país, em porcentagem. É importante dizer 
que, como fonte, a página oficial da OMT utiliza as informações vindas dos países por 
questionários estatísticos anuais, ademais os limites, nomes e designações utilizadas 
no mapa não implicam indicações oficiais ou aceitas pela OMT. Outrossim, a Romênia 
possuía em 2021 a porcentagem de 1,58\% do PIB geral voltado somente para o turismo 
e de 2,4\% na Holanda.  

\begin{figure}[H]
    \centering
    \includegraphics[width=1\linewidth]{image-5.png}
    \caption{Fonte: Dados oficiais de turismo.\url{https://www.untourism.int/tourism-statistics/tourism-data-macroeconomic-indicators}. Acesso em 23 de nov de 2025.}
    \label{fig:placeholder}
\end{figure}

\begin{figure}[H]
    \centering
    \includegraphics[width=1\linewidth]{image-7.png}
    \caption{Fonte: Dados oficiais de turismo. Disponível em: \url{https://www.untourism.int/tourism-statistics/tourism-data-macroeconomic-indicators}. Acesso em 23 de nov de 2025.}
    \label{fig:placeholder}
\end{figure}

Analisando todos esses fatores percebemos que existe uma influência econômica da 
atividade turística nesses países, que geram renda per capta e são importantes 
indicadores. 



% ----------------------------------------------------------
% resultados
% ----------------------------------------------------------
\chapter{Considerações Finais}


 O presente estudo culmina com a consolidação da análise sobre os impactos econômicos do turismo literário, tomando o vilarejo de Bran, na Romênia, como um emblemático estudo de caso. A associação do Castelo de Bran ao mito de Drácula transcende a esfera cultural, materializando-se em uma complexa dinâmica econômica que exige uma avaliação rigorosa, ancorada na distinção e interconexão entre os domínios macro e microeconômico. A perenidade e o desenvolvimento deste destino turístico estão intrinsecamente vinculados à gestão eficaz dessas duas esferas.

A macroeconomia exerce uma influência determinante sobre Bran, estabelecendo o quadro geral no qual a atividade turística se insere. Os dados e as conclusões do projeto se relacionam com a macroeconomia na medida em que o sucesso do turismo em Bran é condicionado por variáveis de nível nacional e internacional. Por exemplo, a análise da origem dos turistas e do volume de gastos (dados do projeto) deve ser interpretada à luz da estabilidade econômica global e do desempenho do Produto Interno Bruto (PIB) romeno (macroeconomia). A estabilidade econômica e a inserção em blocos comerciais são fatores que influenciam a decisão de viagem e o poder de compra dos visitantes. De forma análoga, os investimentos em infraestrutura de transporte e as campanhas de promoção turística nacional (políticas macroeconômicas) são pré-requisitos para que o fluxo de visitantes (dado do projeto) se concretize. Em essência, a macroeconomia define o potencial máximo de crescimento do turismo em Bran, e os dados do projeto servem como indicadores de desempenho que medem o quão bem o destino está capitalizando esse potencial.

Em contrapartida, a microeconomia constitui o nível de análise onde os impactos diretos do turismo se manifestam e onde as decisões individuais dos agentes econômicos são tomadas. Os dados do projeto como a taxa de ocupação hoteleira e o volume chegada de turistas, são evidências diretas da microeconomia em ação. A demanda turística, impulsionada pelo atrativo literário, permite que os prestadores de serviços locais (hotéis, restaurantes, comércio) exerçam um poder de precificação que maximiza seus lucros. A microeconomia, portanto, traduz o potencial macroeconômico em resultados concretos para a comunidade, evidenciados pela alocação de recursos, pela criação de empregos e pelo fomento ao empreendedorismo especializado. A análise microeconômica dos dados do projeto permite avaliar a eficiência e a competitividade dos negócios individuais dentro do vilarejo, revelando como a estrutura de custos e a margem de lucro são afetadas pela sazonalidade e pela concorrência local.

A importância de compreender e definir essas esferas econômicas é fundamental para a validade da análise econômica do turismo, conforme Dwyer, Forsyth e Spurr (2004). Os autores sublinham que a análise dos impactos econômicos do turismo é a evidência mais tangível e convincente da contribuição do setor para o desenvolvimento regional. A distinção clara entre macro e microeconomia, portanto, permite que a análise dos dados do projeto vá além da simples contagem de receitas, incorporando a perspectiva do equilíbrio geral e considerando os custos de oportunidade. Adotar essa visão metodológica confere maior credibilidade às conclusões aqui tomadas.

Para enriquecer a contextualização do turismo literário e de memória, o caso da Casa de Anne Frank, em Amsterdã, oferece um contraponto metodológico e temático relevante.  Economicamente o mercado turístico tem uma porcentagem semelhante tanto na Holanda quanto na Romênia comparando-se à proporção da economia geral, se 
diferenciando pelo maior número de entrada e de receita turística no território 
holandês. Levando em consideração que ambos os países possuem uma renda geral 
diretamente conectada com a mineração e exploração de insumos naturais, a 
economia do turismo é importante, mas o governo Romeno pode ser mais atencioso 
com esse mercado. É perceptível que não ocorreu ainda esse investimento 
mercadológico devido a ideologia política e econômica da Romenia, ainda estando nos 
passos iniciais para tirar um proveito mais benéfico a partir desse meio, visto que tem 
grande potencial de ocorrência. Diferente do governo Holandês, que já possui, de 
acordo com o site da Casa-Museu de Anne Frank, grandes iniciativas de 
aproveitamento do mercado turístico para benefícios em outros indicadores. O nosso 
trabalho demonstra, após diversos infográficos e números oficiais encontrados, a 
influência e impacto do indicador econômico para o turismo desses territórios.  

Em termos de trabalhos futuros, a presente pesquisa sugere diversas propostas de investigação para aprofundar a compreensão do tema. Recomenda-se a investigação da elasticidade-preço da demanda dos turistas de Bran e a realização de um estudo comparativo de competitividade com outros destinos temáticos (para além do aqui estudado), utilizando o modelo de Dwyer e Kim (2003). Finalmente, a avaliação dos impactos sociais e ambientais do turismo em Bran também é uma porta de oportunidade, para assim garantir que o desenvolvimento econômico se harmonize com a qualidade de vida dos residentes e a preservação do patrimônio. 




% ----------------------------------------------------------
% REFERÊNCIAS
% ----------------------------------------------------------
\postextual
\renewcommand{\refname}{Referências}

\begin{thebibliography}{99}

\bibitem{Santos2020} SANTOS, Aremis. Dark Tourism: O caso do Diário de Anne Frank e sua relação com o turismo de memória. 2020.

\bibitem{OECD2016} ORGANIZAÇÃO PARA A COOPERAÇÃO E DESENVOLVIMENTO ECONÔMICO (OCDE). Relatório sobre o Impacto do Turismo na Economia de Países Membros. 2016.

\bibitem{CasaAnneFrank} ANNE FRANK HOUSE. Relatório Anual de 2024. Disponível em: \href{https://www.annefrank.org/en/about-us/annual-report-2024/}{https://www.annefrank.org/en/about-us/annual-report-2024/}. Acesso em: 24 nov. 2025.

\bibitem{UNTourismIndicators} UN TOURISM. Tourism Statistics: Macroeconomic Indicators. Disponível em: \href{https://www.untourism.int/tourism-statistics/tourism-data-macroeconomic-indicators}{https://www.untourism.int/tourism-statistics/tourism-data-macroeconomic-indicators}. Acesso em: 24 nov. 2025.

\bibitem{baleiro2018} BALEIRO, R.; QUINTEIRO, S. \textbf{Literary Tourism: An Interdisciplinary Approach}. Channel View Publications, 2018. Disponível em: \url{https://www.channelviewpublications.com/display.asp?k=9781845416726}.

\bibitem{brasov2025} DIRECȚIA JUDEȚEANĂ DE STATISTICĂ BRAȘOV. \textit{Investiții}. Disponível em: \url{https://brasov.insse.ro/comunicate-de-presa/investitii/investitii/#respond}. Acesso em: 28 out. 2025.

\bibitem{brasov2021} DIRECȚIA JUDEȚEANĂ DE STATISTICĂ BRAȘOV. \textit{Recensământul Populaţiei şi Locuinţelor 2021}. Disponível em: \url{https://brasov.insse.ro/produse-si-servicii/recensaminte/}. Acesso em: 28 out. 2025.


\bibitem{butler2012} BUTLER, R. \textbf{The concept of a tourist area cycle of evolution: implications for management of resources}. Canadian Geographer, 2012. Disponível em: \url{https://www.researchgate.net/publication/228003384_The_Concept_of_A_Tourist_Area_Cycle_of_Evolution_Implications_for_Management_of_Resources}.

\bibitem{castelobran} BRAN CASTLE. \textit{Official Bran Castle website}. Disponível em: \url{https://www.bran-castle.com/en/}. Acesso em: 21 nov. 2025.

\bibitem{cechinel2016} CECHINEL, Anderson; ALBINO, Jéssica; SEBASTIÃO, Liliane. \textit{Análise documental e pesquisa qualitativa: uma aproximação metodológica}. Anais do Congresso Nacional de Educação, Curitiba, 2016.

\bibitem{cellard2008} CELLARD, André. \textit{A análise documental}. In: POUPART, Jean et al. (org.). \textit{A pesquisa qualitativa: enfoques epistemológicos e metodológicos}. Petrópolis: Vozes, 2008. p. 295–316.

\bibitem{comanescu2023} COMĂNESCU, Adrian Şerban. \textit{The Direct Contribution of Tourism to the Gross Domestic Product of Romania}. EasyChair Preprint, 1251. Recuperado de: \url{https://easychair.org/publications/preprint/1251}.

\bibitem{constantino2019} CONSTANTINO, A. \textit{Fluxos turísticos entre os países do Corredor Bioceânico}. \textit{Interações}, v. 20, n. 2, p. 123-145, 2019. Disponível em: \url{https://www.scielo.br/j/inter/a/JZpXf4RfYwrpmJWCWpVLHCB/?lang=pt}. Acesso em: 28 out. 2025.

\bibitem{dwyer2003} DWYER, L.; SPURR, R.; KIM, C. \textit{Destination Competitiveness: Determinants and Indicators}. Journal of Travel Research, v. 42, n. 4, p. 305–314, 2003. DOI: \url{https://doi.org/10.1177/0047287503257533}.

\bibitem{dwyer2004} DWYER, L.; FORSYTH, P.; SPURR, R. \textbf{Evaluating tourism’s economic effects: new and old approaches}. Tourism Management, v. 25, n. 3, p. 307–317, 2004. Disponível em: \url{https://doi.org/10.1016/S0261-5177(03)00131-6}.

\bibitem{fletcher2018} FLETCHER, J. et al. \textbf{Tourism: Principles and Practice}. Pearson, 2018. Disponível em: \url{https://www.sciencedirect.com/science/article/abs/pii/S0261517703001316}.

\bibitem{frost2016} FROST, W. \textbf{Turismo e Balcanismo, a partir do Dracula de Bram Stoker}, v. 57, p. 45–58, 2016. Disponível em: \url{https://www.academia.edu/27446060/Turismo_e_balcanismo_a_partir_do_Drácula_de_Bram_Stoker}.

\bibitem{gil2010} GIL, Antonio Carlos. \textit{Como elaborar projetos de pesquisa}. 5. ed. São Paulo: Atlas, 2010.

\bibitem{ibge2003} BRASIL. Ministério do Planejamento, Orçamento e Gestão; Instituto Brasileiro de Geografia e Estatística (IBGE). \textit{Economia do turismo: análise das atividades características do turismo – 2003}. Rio de Janeiro: IBGE, 2006. 62 p. (Estudos e Pesquisa. Informação Econômica, n. 5). ISBN 85-240-3923-X. Disponível em: \url{https://www.gov.br/turismo/pt-br/acesso-a-informacao/acoes-e-programas/observatorio/repositorio/economia-do-turismo/economia_turismo___dados_de_2003.pdf}. Acesso em: 27 out. 2025.

\bibitem{lima2021} LIMA JUNIOR, José Alves; OLIVEIRA, Rita de Cássia; SANTOS, Marisa; SCHNEKENBERG, Marilene. \textit{Análise Documental como percurso metodológico na pesquisa qualitativa}. Revista FUCAMP, v. 24, n. 47, p. 12–25, 2021. Disponível em: \url{https://revista.fucamp.edu.br/index.php/revistafucamp/article/view/2742}.

\bibitem{ludke1986} LÜDKE, Menga; ANDRÉ, Marli Eliza Dalmazo Afonso de. \textit{Pesquisa em educação: abordagens qualitativas}. São Paulo: EPU, 1986.

\bibitem{reijnders2011} REIJNDERS, S. (2011). \textit{Stalking the Count: Dracula, Fandom \& Tourism}. \textit{Annals of Tourism Research}, 38(1), 231--248. Disponível em: \url{https://doi.org/10.1016/j.annals.2010.08.006}.

\bibitem{ribeiro2021} RIBEIRO, Luiz Carlos de Santana; SANTOS, Monique Manuela Carvalho dos; SANTOS, Fernanda Rodrigues dos. \textit{Avaliação das Atividades Características do Turismo no Brasil: 2012–2020}. \textit{Turismo: Visão e Ação}, v. 23, n. 3, p. 557–578, 2021. DOI: \url{https://doi.org/10.14210/rtva.v23n3.p557-578}.

\bibitem{santana2018} SANT ANA, W. P.; LEMOS, G. C. (2018). \textit{Metodologia científica: a pesquisa qualitativa nas visões de Lüdke e André}. \textit{Revista Eletrônica Científica Ensino Interdisciplinar}. Disponível em: \url{https://periodicos.apps.uern.br/index.php/RECEI/article/view/1710/1669}.

\bibitem{smith1996} SMITH, Adam. \textit{A riqueza das nações: investigação sobre sua natureza e suas causas}. São Paulo: Nova Cultural, 1996.

\bibitem{tuzzo2016} TUZZO, Simone Amado; BRAGA, Rosane de Fátima. \textit{Triangulação e pesquisa em comunicação}. Revista FAMECOS, v. 23, n. 3, p. 1–15, 2016. DOI: \url{https://doi.org/10.15448/1980-3729.2016.3.23314}.

\bibitem{WTTC2023} World Travel and Tourism Council (WTTC). (2023). The Economic Impact of Travel \& Tourism 2023: Romania. Recuperado de \url{https://wttc.org/Research/Economic-Impact}.

\bibitem{Eurostat2023} Eurostat. (2023). Tourism Statistics - Romania. Recuperado de \url{https://ec.europa.eu/eurostat/web/tourism}.


\end{thebibliography}



\end{document}
